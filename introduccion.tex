\chapterimage{imagenes/introduccion.jpg}
\chapterimagedescription{Imagen de la supercomputadora Thunderbird, en el Sandia National Laboratory.}
\chapterimageauthor{Fotografía del Departamento de Energía de los Estados Unidos}

\chapter*{Introducción}
\label{intro}

La \textbf{informática} es la disciplina o campo de estudio que abarca el
\textbf{conjunto de conocimientos, métodos y técnicas referentes al tratamiento
automático de la información, junto con sus teorías y aplicaciones prácticas,
con el fin de almacenar, procesar y transmitir datos e información en formato
digital utilizando sistemas computacionales}. Los datos son la materia prima
para que, mediante su proceso, se obtenga como resultado información. Para ello,
la  informática crea y/o emplea sistemas de procesamiento de datos, que incluyen
medios físicos en interacción con medios lógicos y las personas que los
programan y los usan.

\section*{Estructura del libro}

El presente se estructura en cuatro unidades y una unidad adicional que contiene
el índice y una serie de anexos. Cada unidad a su vez, se divide en capítulos, y
cada capítulo en diferentes secciones.

Cada capítulo del presente se apoya en lo visto en capítulos anteriores, y si
bien pueden leerse de forma independiente, esto trae aparejado el supuesto de
ciertos conocimientos ya vistos, por lo que recomendamos leer el libro de forma
completa y continuada.

El texto intenta ser lo más breve y directo posible, para permitir la
comprensión incluso a los menos asiduos a la lectura. Por otro lado, se intenta
explicar los conceptos con definiciones más pragmáticas que enciclopédicas. Si
bien esto puede no ser del agrado de todos los lectores, creemos que la
comprensión del concepto es más importante que una definición, la cual, en
general, varía de autor en autor. Si bien en algunos lugares se sacrifica
precisión en pos de la comprensión, en todo lugar donde esto ocurre es
deliberado, pues no se desea sobrecargar el libro en conceptos que, si bien
importantes para un profesional de la industria, no son relevantes para quien
busca adquirir conocimientos generales de la temática o recién está iniciando su
camino como profesional.

Al final de cada capítulo se presentan ejercicios prácticos a realizar, los
cuales tienen dos propósitos. Por un lado, muchos de ellos sirven para asentar
los conocimientos teóricos vistos utilizando un enfoque práctico. Por otro,
algunos de ellos ocultan detalles teóricos interesantes que no pudieron entrar
en estas páginas por motivos de espacio.

Los ejercicios suelen tomar desde unos pocos minutos a una hora en resolverse, y
permiten validar al lector si ha adquirido los conocimientos tratados en el
capítulo, así como expandir los saberes presentados en el mismo. Recomendamos
que se vayan realizando a medida que se avanza con la lectura.

Al final de cada unidad se presenta un conjunto de bibliografía adicional que
permitirá al lector expandir sus conocimientos o abordar con mayor profundidad
las diversas temáticas encaradas aquí, así como también tener otros puntos de
vista de diferentes autores.

La última parte del libro contiene el índice con las palabras clave, y los
lugares en donde dichas palabras aparecen en el libro de forma relevante.
También posee una serie de anexos, los cuales presentan información adicional
sobre una temática particular, o resultan útiles de tener a mano a la hora de
realizar las actividades propuestas.

\section*{Temas a tratar}

Como este campo requiere el uso de \textbf{computadoras} (en el sentido amplio
de la palabra), comenzaremos en la \autoref{unit:computadoras} por analizar qué
es una computadora, ver como funcionan, de que partes están compuestas y una
breve historia de las mismas. Se debe tener en cuenta que nos centraremos en
computadoras electrónicas, también llamadas clásicas, las cuales son las que
predominan en el mercado. Sin embargo, analizaremos otras variantes de
computadoras cuando veamos la historia de las mismas, como las computadoras
mecánicas. Asimismo, los conceptos teóricos que rigen las ciencias de la
computación, auguran como posibilidades otros modelos computacionales, todavía
en etapas muy inmaduras, como las computadoras cuánticas o las biológicas, tema
que tocaremos brevemente. Este capitulo busca brindar al lector una introducción
amena a cuáles son los elementos de estudio pertinentes, entender cómo llegaron
a existir, su importancia en nuestra vida cotidiana actual, y también tener una
mínima visión sobre su posible futuro. El enfoque de esta unidad es más bien
informativo, y las actividades finales solo expanden sobre diversos elementos de
la historia y conocimientos generales.

En la \autoref{unit:informacion} analizaremos como las computadoras almacenan
\textbf{información}, y como se procesa y se interpreta la misma. Haremos una
mínima introducción a lo que es el concepto de datos binarios, y veremos como
con ese sistema se almacenan números, letras, colores, imágenes, videos y
cualquier información que uno desee. Veremos como funcionan los editores de
texto y los visualizadores de archivos, y tendremos un primer acercamiento a un
lenguaje informático mediante el uso de lenguajes de marcado. Evitaremos
mencionar los estándares que rigen las computadoras modernas, pues solamente nos
interesará el concepto subyacente a los mismos, por lo que algunos lectores que
posean conocimiento de esta temática podrán sentir que se ha tomado una
aproximación muy laxa, pero recuerde el lector que el objetivo final de esta
sección es dar una idea mínima sobre el funcionamiento de las computadoras
modernas, para que el mismo no resulte ``mágico''. Si bien el enfoque de esta
unidad es también informativa, las actividades finales que se proponen implican
reforzar los elementos conceptuales mencionados para un mayor entendimiento.

Posteriormente en la \autoref{unit:logica} analizaremos la \textbf{lógica}, uno
los fundamentos teóricos más importantes que subyace en la disciplina.
Analizaremos parte de los formalismos que rigen en esta ciencia y veremos como
se aplican esos conceptos en matemática y en ciencias de la computación en
general. El foco del presente no es realizar un análisis exhaustivo de esta
disciplina, la cual es sumamente amplia para abarcarla en tan pocas páginas,
sino brindar los conocimientos elementales que permitirán al lector aplicar
estos conceptos a diferentes áreas, como matemáticas, programación e incluso en
su día a día. Aparecerán entonces los conceptos de conectivas, y las
utilizaremos para formular preguntas nuevas a partir de otras que ya teníamos
disponibles, un concepto muy útil a la hora de programar. También hablaremos de
razonamientos deductivos, de sus premisas y conclusión, cuándo son válidos y
cuándo no, y aplicaremos los conocimientos de forma práctica para introducirnos
en la lógica proposicional y descubrir sus formalismos. Finalmente, veremos
razonamientos que no pueden resolverse con la lógica proposicional y mojaremos
nuestros pies en la superficie de la lógica de predicados, analizando sus
elementos, aunque no trabajaremos con el análisis de la validez de dichos
razonamientos. Esta unidad es más bien teórico-práctica, ya que, de forma
similar a la matemática, si bien se pueden comprender los conceptos de la lógica
fácilmente, su aplicación posterior resulta compleja si no se ha expuesto uno a
varias actividades de la temática. Así, se recomienda realizar cada uno de los
ejercicios propuestos en cada unidad, ya que su realización es fundamental para
un entendimiento adecuado de la temática.

Por último, en la \autoref{unit:programacion} veremos como se procesa
información utilizando \textbf{programación}. Crearemos nuestros primeros
programas que solucionarán pequeños problemas, y analizaremos los mismos para
descubrir algunas de las estructuras más comúnmente utilizadas en programación.
Nuevamente, la disciplina es amplia, y por tanto solo abordaremos actividades
muy simples, que incluyen temas sencillos, como la secuencia de comandos para
generar programas, el uso de estructuras de control para estructurar ideas y
reducir la cantidad de código, así como generar soluciones más genéricas. Por
sobre todo trabajaremos en planificar nuestras soluciones en forma de tareas
pequeñas para simplificar la complejidad del problema a resolver, así como
comunicar correcta y eficazmente nuestra solución. Los lectores que tengan
conocimientos previos en esta área sentirán al comienzo que lo presentado es muy
básico, y que aporta poco a su saber. Sin embargo, recomendamos no tomarse a la
ligera lo presentado en esta unidad, pues el abordaje elegido difiere del
seleccionado por muchos autores, y hace especial hincapié en la comunicación de
la solución al problema, concepto fundamental para convertirse en un programador
profesional. Esta unidad tiene un enfoque bastante distinto a las anteriores, en
donde se propone al lector realizar actividades a medida que se avanza en la
lectura, para luego analizar diversas soluciones al problema, y a partir de ahí
generar el conocimiento. Es recomendable que el lector encare dichas actividades
en el momento sugerido, poniendo en pausa la lectura hasta terminar las
actividades propuestas, y retomando solo despues de haber pensado la solución y
haberla finalizado. Las actividades propuestas deben realizarse en computadora,
tablet o teléfono inteligente (aunque en este último la experiencia no es la
mejor), por lo que se recomienda que el lector cuente con acceso a uno de estos
dispositivos durante la lectura de esta última unidad, aunque sea de forma
esporádica. Adicionalmente elaboraremos una sintaxis para escribir programas en
papel, algo que permitirá realizar las actividades incluso si no se tiene acceso
a computadora, y que además será útil para una mejor comprensión de los
conceptos trabajados, que se reforzarán al final de cada capítulo con mayor
cantidad de actividades.

\section*{Requisitos y saberes previos}

Para poder comprender los contenidos de este libro no se requieren demasiados
saberes previos. Se espera sin embargo, que el lector posea algunos conceptos
básicos de matemática, que incluyen aritmética elemental y algo de geometría.

Saberes elementales de la lengua también son útiles, como poder comprender
textos, analizar sujeto y predicado, identificar verbos, y poder determinar
donde hay un sujeto tácito o se hace referencia a un sujeto antes mencionado.

También se espera que el lector sepa utilizar una computadora, o se haya cruzado
en algún momento con una. Si bien existen diversos tipos de computadora, y
diversas formas de interactuar con las mismas, el presente toma por supuesto que
el lector ha utilizado una computadora de escritorio o notebook, y que sabe lo
que es un teléfono móvil inteligente (smartphone), haya o no interactuado con
uno.

La \autoref{unit:logica}, en su última sección, se vuelve más sencilla de
comprender si el lector se cruzó previamente con el concepto de función (en
términos de análisis matemático), aunque no es imprescindible para la correcta
comprensión de los temas abordados.

Para seguir la teoría de la \autoref{unit:programacion}, así como resolver los
ejercicios propuestos puede recurrir solamente a la lectura, lápiz y papel. Sin
embargo, realizar los ejercicios en la computadora facilitará una mayor
comprensión de los conceptos, simplificando el proceso de aprendizaje. Para esto
se espera que lector cuente con acceso a una computadora, ya sea una notebook,
máquina de escritorio, tablet o incluso un teléfono inteligente y acceso a
internet (ya sea continuo o esporádico).

\section*{Comentarios finales}

La disciplina de las ciencias de la computación y la informática en general son
campos relativamente nuevos (aunque sus orígenes datan de más de tres milenios
atrás, no se comenzó a considerar la informática como una ciencia en si misma
sino hasta comienzos del siglo XX). Aún así, esta disciplina ha evolucionado
mucho en un muy corto período de tiempo, revolucionando de forma sustancial el
mundo en el que vivimos.

Si bien las computadoras son omnipresentes en nuestro día a día, todo indica que
recién estamos nadando en la superficie de las aguas de lo que es posible con la
informática, y que el futuro nos depara una inmersión en las mismas, a
profundidades que aún no alcanzamos a vislumbrar. Las posibilidades auguradas
por las ciencias de la computación son muchas y muy amplias, y es un campo en
completa expansión, que trabaja cada vez más de forma interdisciplinar con
ciencias tan dispares como la física, la biología, la neurología, la
lingüística, la economía, la política, e incluso la filosofía.

Por otro lado, como ya mencionamos, la informática tiene que ver con el proceso
y manejo de la información. Ser meros usuarios de tecnologías, desconociendo
completamente los conceptos subyacentes más básicos de los mismos, nos vuelve
vulnerables a engaños y manipulaciones por quienes controlan la información. ¿Es
bueno el voto electrónico? ¿Qué datos tienen las empresas sobre mi persona y
cómo pueden utilizarlos? ¿Es seguro utilizar una tarjeta de crédito en internet?
Si conocemos los conceptos subyacentes a la tecnología que utilizamos, responder
esas preguntas se vuelve más sencillo, y nos permite tomar mejores decisiones
como ciudadanos digitales.

Muchos especialistas afirman que dentro de pocos años en el futuro, desconocer
estos conceptos será equivalente a no saber leer y escribir en la sociedad
actual. Por eso este libro apunta a brindar un panorama introductorio, pero lo
suficientemente amplio como para que el lector obtenga un panorama general de
los conceptos principales de la informática generalmente desconocidos.
