\newcommand{\apple}{\emoji{unidades/3_logica/1_logica/imagenes/apple.pdf}}
\newcommand{\orange}{\emoji{unidades/3_logica/1_logica/imagenes/orange.pdf}}

\chapterimage{unidades/3_logica/1_logica/imagenes/cover}
\chapterimagedescription{Red Lógica de un Enrutador con LEDs}
\chapterimageauthor{Fotografía de tschoenemeyer}

\chapter{Introducción a la lógica}
\index{Introduccion a la logica@Introducción a la lógica}
\label{chap:logica}

Construimos las computadoras para facilitarnos el realizar cáclulos, algo que
nos ha caracterizado como especie por milenios. Pero hacer cuentas no es la
única característica propia de los seres humanos. Somos, al menos por lo que
sabemos, la única especie capaz de producir información novedosa mediante el
pensamiento y la razón.

La \textbf{razón}\index{Razon@Razón} es la base de la ciencia. Es lo que nos ha
permitido formular teorémas, leyes, postulados, etc. y es lo que permite que
todas las disciplinas y ciencias, tanto naturales, empíricas, formales, sociales
o económicas, avancen constantemente. Aunque no solo aplica a la ciencia, pues
razonamos de forma constante. Con nuestra mente detectamos patrones, realizamos
conexiones entre información y llegamos a conclusiones sobre lo que sucedería en
casos hipotéticos. Esto ha permitido a nuestra especie tomar decisiones que nos
ayudan a preservar nuestra vida, ser más competitivos con respecto a otros,
consiguiendo más y mejores recursos, y por tanto, evolucionar con ventaja por
sobre otras especies.

La \textbf{lógica}\index{Logica@Lógica} es el estudio de esos procesos mentales
que llevan a la razón. Es una ciencia compleja, con diversos enfoques, muchisima
historia y ramificaciones.

Siendo la lógica una ciencia en si misma, su estudio completo abaracaría mucho
más que las escasas páginas que le dedicaremos en esta unidad, mucho más que un
único libro de hecho. Aquí solamente nos adentraremos en algunos conceptos
fundamentales, que dan sustento a gran parte de los elementos de las ciencias de
la computación, y que luego aplicaremos en la sección de programación, pero que
también aparecerán en el futuro en múltiples lugares para quienes opten por
dedicarse al estudio de alguna disciplina relacionada a las ciencias de la
computación y la informática.

En esta unidad comenzaremos por entender la lógica desde los conocimientos
previos que se espera el lector posea. Si bien indagaremos en definiciones, no
entraremos en profundos formalismos, ni presentaremos un sistema lógico formal
de análisis, dejando estos elementos para los siguientes capítulos.

\section{Aproximación a la lógica}
\label{chap:logica:sec:aproximacion}

Antes de empezar a trabajar con la lógica como ciencia formal, empecemos por
tener una aproximación sobre esta ciencia que, incluso sin saberlo, utilizamos
de forma constante en nuestro día a día. Vamos a abordar primero de dónde surge,
y luego, cuáles son algunos de los elementos constituyentes claves. A posteriori
veremos por qué es útil conocer y entender de lógica para un programador.

Comencemos pues por el principio, de dónde surge la lógica.

\subsection{Historia de la lógica}
\label{chap:logica:subsec:historia_logica}

La idea de razonar es algo que define a la especie humana, y por tanto no es
de extrañar que ya en la Antigua Grecia alrededor del Siglo IV a.C. los
filósofos intentaran comprender cómo funciona el proceso del pensamiento y la
razón. La palabra actual ``lógica'' deriva del griego antiguo \textit{logiké}
($\lambda o \gamma \iota \kappa \eta$), que significa ``\textit{dotada de razón,
intelectual, argumentativa}''  y que a su vez viene de \textit{lógos} ($\lambda
o \gamma o \varsigma$), ``\textit{palabra, pensamiento, idea, argumento, razón o
principio}''. Así, la palabra ``\textit{lógica}'' se refiere a la ciencia que
tiene por estudio el pensamiento y la razón. Por supuesto, los griegos no fueron
los únicos en realizar estudios sobre el tema, ya que tanto en China como en la
India aparecieron de forma independiente tratados sobre el tema alrededor de la
misma época que en Grecia.

Aristóteles\index{Aristoteles@Aristóteles} fue el primero en formalizar los
razonamientos, utilizando letras para representar términos y estableciendo las
fórmas válidadas de los mismos. Además estudió los razonamientos inductivos,
base de lo que constituye la ciencia experimental, cuya lógica está
estrechamente ligada al método científico. Los islámicos, y luego los lógicos de
la Edad Media continuarían el analisis aristotélico de la lógica, vinculado más
a la filosofía que a otras ciencias.

\wraplimage[0.3]{unidades/3_logica/1_logica/imagenes/aristoteles_ceuta.jpg}
{Busto de bronces de Aristóteles en la ciudad de Ceuta, España, del artista
Serrán Pagán.} {Fotografía de Wikimedia Commons de CarlosVdeHasburgo}

Ya en la modernidad, se comenzó a estudiar la lógica desde puntos de vista más
formales, y vincular la ciencia a las matemáticas. Matemáticos como Gottlob
Frege\index{Frege, Gottlob}, Giuseppe Peano\index{Peano, Giuseppe}, Ernst
Schröder\index{Schroder, Ernst@Schröder, Ernst} y Charles Peirce\index{Pierce,
Charles} extendieron los conceptos de la lógica aristotélica, agregando
elementos como las variables o los cuantificadores ``todo'' y ``algunos'',
permitiendo revelar la estructura lógica de las oraciones en lenguajes
naturales. Así, se iría extendiendo el estudio de esta ciencia, definiendo
``\textit{lógicas}'' de distintos ``\textit{ordenes}'' (primer y segundo orden).

Por su parte , George Boole\index{Boole, George}, a quien ya hemos mencionado en
este libro, publicó un breve tratado titulado ``\textit{El análisis matemático
de la lógica}'', y en 1854 otro más importante titulado ``\textit{Las leyes del
pensamiento}''. La idea de Boole fue construir a la lógica como un cálculo con
propiedades similares a la matemática. Al mismo tiempo, Augustus De
Morgan\index{De Morgan, Augustus} publica en 1847 su obra ``\textit{Lógica
formal}'', donde introduce las ``\textit{leyes de De Morgan}'' e intenta
generalizar la noción de silogismo. Otro importante contribuyente fue John Venn,
quien en 1881 publicó su libro ``\textit{Lógica Simbólica}'', donde introdujo
los famosos diagramas de Venn.

Ya en 1910, a raíz de varias crisis en las bases fundamentales de la matemática,
Bertrand Russell\index{Rusell, Bertrand} y Alfred Whitehead\index{Whitehead,
Alfred} publican ``\textit{Principia mathematica}''\index{Principia
Mathematica}, un trabajo en el que logran construir gran parte de las
matemáticas a partir de la lógica, demostrando que la lógica es una ciencia
subyacente a las matemáticas.

Más de 2000 años de historia nos llevan a la actualidad de la lógica como
ciencia, que hoy sustenta cualquier desarrollo científico, tanto de las ciencias
empíricas (biología, química, física, etc.) como formales (matemáticas y sus
diversas ramas) y también las ciencias sociales (legales, comunicación,
filosofía, etc.) y económicas. Hoy en día se entiende a la lógica como una
ciencia que requiere un análisis formal (parecido a las matemáticas, como
veremos luego) y que sigue métodos estrictos para su análisis.

Existen diversos ``\textit{típos de lógicas}'', como la lógica proposicional, la
lógica de predicados, la logica hermitica, logicas modales, entre otras varias.
También el enfoque de estudio varía, según la necesidad de la disciplina
subyacente. Para las ciencias de la computación nos interesará centrarnos en la
lógica proposicional y la lógica de predicados. Pero necesitamos entender mejor
qué es la lógica, como ciencia en general, antes de poder estudiar cualquiera de
ellas.

\subsection{El objeto de estudio de la lógica}
\index{Lógica}\index{Razonamientos}\index{Deducción}
\label{chap:logica:subsec:objeto_de_estudio}

Para poder comprender la \textbf{lógica} debemos comenzar por saber qué es y qué estudia. Para ello, veremos primero una definición simple.

\begin{definition}\index{Logica@Lógica} La \textbf{lógica} es una ciencia formal
    que estudia los principios de la deducción de los razonamientos, así como su
    validez.
\end{definition}

Pero para entender esta definición, primero debemos comprender qué es un
\textbf{razonamiento}. Veamos una segunda definición para ello.

\begin{definition}\index{Razonamiento}\index{Premisa}\index{Conclusion@Conclusión}
    Un \textbf{razonamiento} es el producto de un \textbf{proceso mental}
    mediante el cual, a partir de una o más piezas de información de la cual se
    dispone, llamadas \textbf{premisas}, se arriba a una nueva pieza de
    información, llamada \textbf{conclusión}.
\end{definition}

Es decir, el \textbf{razonamiento} es el producto de
\textbf{razonar}\index{Razonar}, un proceso que produce el cerebro humano. En
este proceso tomamos información previamente disponible, es decir, información
ya conocida, para descomponerla, analizarla y procesarla, y arribar así a una
nueva información.

Lo más fácil para comprender el concepto es ver un ejemplo concreto, por lo que
recurriremos a un ejemplo clásico. Considere la siguiente información como
\textbf{premisas}, es decir, como información ya conocida a priori.

\begin{example}
    \textit{Todos los humanos son mortales}.\\
    \textit{Socrates es un humano}.
\end{example}

Una \textbf{conclusión} posible sería determinar que ``\textit{Socrates es
mortal}''. Nuestro cerebro puede realizar el analisis de las dos primeras
oraciones y encontrar que hay elementos relacionados entre ellas, elucidando una
información que no era conocida previamente (el hecho de que Socrates sea
mortal).

Notar que si alguien pensara como conclusión que ``\textit{Socrates es
inmortal}'', e intentara atravesarlo con una espada para probarlo, probablemente
intentaríamos impedirlo inmediatamente. Es que de la información contenida en
las premisas no hay forma de concluir la inmortalidad de Socrates, salvo
mediante un proceso mental incorrecto. Así, la persona que arribo a dicha
conclusión, no ha razonado de forma adecuada, llegando a una conclusión que no
tiene sentido.

\textbf{Los razonamientos pueden partir de una o más premisas, pero siempre
tienen una única conclusión}. A veces nuestro cerebro realiza múltiples
conclusiones a partir de un conjunto de premisas, pero con respecto a lo que
concierne a la lógica estos se consideran múltiples razonamientos, y se debe
estudiar cada uno de forma independiente.

La \textbf{deducción}\index{Deducción} es entonces el proceso que realizamos
para partir de información conocida para arribar a nueva información. La
deducción sigue formas bien definidas y precisas, y es por eso que podemos
concluir en la mortalidad de Socrates, pero no en su inmortalidad. Si concluimos
lo segundo, nuestro razonamiento sería \textbf{inválido}, mientras que por el
contrario, si concluimos lo primero, podemos decir que es un razonamiento
\textbf{válido}.

La lógica se va a encargar de estudiar entonces cuándo un razonamiento es
válido, y cuándo no lo es.

\begin{knowwhat}\index{Razonamiento deductivo}
    \index{Razonamiento inductivo}
    En realidad, nos estamos centrando en un tipo particular de razonamientos,
    llamados \textbf{razonamientos deductivos}, y por tanto estamos abusando de
    la simplificación en nuestras definiciones.

    Existen otros tipos de razonamientos, como los inductivos (que surgen de
    razonar por inducción, en lugar de por deducción), y que pueden ser a su vez
    inductivos por enumeración, o inductivos por analogía. En el primero de este
    tipo de razonamientos se obtiene una conclusión global a partir de premisas
    que hablan de casos particulares (Por ej. de las premisas ``\textit{la droga
    $X$ resulto exitosa para curar la enfermedad en el paciente 1}'',
    ``\textit{La droga $X$ resulto exitosa para curar la enfermedad en el
    paciente 2}'', etc. se puede concluir que ``\textit{La droga $X$ resultará
    exitosa para curar a todos los pacientes}''), en el segundo se obtiene un
    caso particular a partir de otros similares (Por ej. con las premisas
    anteriores arribar a ``\textit{La droga $X$ resultará exitosa para curar al
    paciente N}'').

    Este tipo de lógica es útil en las ciencias empíricas, donde a veces, el
    hecho de que se cumpla la conclusión el 99\% del tiempo es lo
    suficientemente bueno. Sin embargo, siempre pueden encontrarse excepciones a
    la regla. En el ejemplo anterior, pueden haber personas que sean alérgicas a
    la droga en cuestión, o que la droga no haga efecto en sus organismos.

    Los razonamientos deductivos, en cambio, tienen una caracteristica
    particular. \textbf{Si las premisas de las que partimos son verdaderas, un
    razonamiento bien realizado siempre arribará a una conclusión verdadera}.
    Esto es así ya que que la información de la conclusión, en realidad está
    ``oculta'' en las premisas, es decir, está expresada en las premisas, aunque
    de otra forma, no evidente. Por eso este tipo de razonamientos es el
    empleado en las ciencias formales (como en la matemática), para realizar
    \textbf{demostraciones}.
\end{knowwhat}

Con esto tenemos más claro de qué se trata la lógica. Pero ¿A qué nos referímos
con \textbf{ciencia formal}\index{Ciencia formal}? Pues bien, se trata de que el
objeto de estudio no son los elementos en sí, sino las formas de los elementos.
Además de la lógica hay solo otra ciencia formal, la matemática. Veamos un
ejemplo de esta última.

\begin{example}
    \hspace*{14pt} Supongamos que tenemos 4 naranjas y decidimos comer una.
    ¿Cuantas naranjas nos quedan para comer? Esto los saben hasta los niños
    pequeños, nos quedan 3 naranjas.

    Si de pronto tenemos 4 manzanas, y nos comemos una, sería ridiculo pensar
    que debemos aprender algo nuevo para poder saber cuántas nos quedan.

    Así, \textit{la forma} de la operación a realizar no tiene que ver con
    naranjas o manzanas, sino con cantidades, con números. Las matemáticas nos
    enseñan que la forma es ``$4-1=3$'', y una vez aprendemos a solucionar ese
    problema, no nos importan qué sean los elementos.

    Dicho de otra forma:

    \dindent Operación real: ~ 4 \orange ~~ - 1 \orange ~ = 3 \orange \qquad
    Forma: ~ 4 - 1 = 3

    y también:

    \dindent Operación real: ~ 4 \apple ~~ - 1 \apple ~ = 3 \apple \qquad Forma:
    ~ 4 - 1 = 3
\end{example}

La lógica opera de forma similar. Pensemos en las premisas ``\textit{Todos los
perros van al cielo}'' y ``\textit{Firulais es un perro}''. La conclusión sería
que ``\textit{Firulais va al cielo}''. La forma para este razonamiento, es la
misma que para el razonamiento anterior sobre Socrates, que sería algo como
``\textit{Todos los $x$ hacen/son algo}'', ``\textit{$a$ es un $x$}'' por lo que
se puede concluir que ``\textit{$a$ hace/es algo}''.

Por ahora no nos interesa analizar la estructura de un razonamiento en detalle,
sino solamente ver que es posible encontrar una estructura idéntica en distintos
razonamientos, para comprender mejor el concepto de ``\textit{forma}'' al que
nos referímos cuando hablamos de lógica.

\section{Elementos constituyentes de la lógica}
\label{chap:logica:sec:elementos_constituyentes}

Una cosa importante a comprender es que no existe ``una lógica'' sino que hay distintos formalismos lógicos, cada una con sus particularidades, propiedades y sus utilidades de aplicación.

\subsection{Valores de verdad}
\index{Valores de verdad}
\label{chap:logica:subsec:valores_de_verdad}

Finalmente, es muy importante mencionar que la lógica se basa en asignar lo que
llamamos un \textbf{valor de verdad} a diversos elementos, de forma similar a
como en las matemáticas se le asignan valores numéricos a los elementos. Esto
permite realizar análisis sobre diversas situaciones para determinar si un
razonamiento es válido o inválido.

Por lo tanto, debemos también entender de qué se trata un \textbf{valor de
verdad}. Empecemos por una definición.

\begin{definition}\index{Valor de verdad} Los \textbf{valores de verdad} son
    valores binarios y dicotómicos, es  decir, dos valores que son opuestos
    entre si, y donde no hay terceras opciones ni respuestas que impliquen ambos
    valores al mismo tiempo.
\end{definition}

En el lenguaje natural, es decir, cuando hablamos en español (u otro idioma),
usamos en general las respuestas ``\textit{si}'' y ``\textit{no}'' como valores
de verdad. Pensemos en una pregunta cualquiera que pueda responderse con si o
no, por ejemplo, ``\textit{¿Está lloviendo el día de hoy en la Ciudad de Buenos
Aires?}''. Según el día y la hora que se esté leyendo esto, la respuesta
concreta será o bien ``\textit{si}'' o bien ``\textit{no}'', pero seguro está
lloviendo o bien no lo está, es decir, la respuesta es binaria. Más aún, los
valores son opuestos, es decir, responder que ``\textit{si llueve}'' es opuesto
a responder que ``\textit{no llueve}'', y no puede darse que llueva y no llueva
al mismo tiempo.

Tenemos otras formas de referirnos a ese ``\textit{si}'' en el lenguaje natural,
por ejemplo ``\textit{afirmativo}'', ``\textit{cierto}'', o
``\textit{verdadero}''. De igual forma, tenemos otras palabras para
``\textit{no}'', como ``\textit{negativo}'', ``\textit{mentira}'' o
``\textit{falso}''.

Al hablar de lógica en términos formales se utilizan los términos
``\textbf{verdadero}''\index{Verdadero} y ``\textbf{falso}''\index{Falso} para
referirnos a los posibles valores de verdad que puede tomar un elemento.

\begin{knowwhat}
Otras disciplinas usar otros nombres para el mismo concepto de valor de verdad.
Por ejemplo, la electrónica utiliza los nombres ``\textit{1}'' y ``\textit{0}''
para este concepto, y lo utilizan para representar la idea de ``\textit{es
verdadero que hay electricidad}'' o ``\textit{es falso que hay electricidad}'',
cómo ya vimos en la unidad anterior.

Hablar de \textit{1} y \textit{0} es confuso para los no iniciados en la
electrónica y en la lógica, ya que esos mísmos símbolos se utilizan en la
matemática, con un significado diferente, numérico.
\end{knowwhat}

La lógica entonces nos permite hablar de elementos verdaderos y falsos, de
formas y estructuras para esos elementos, y de validez o invalidez de las
combinaciones que hacemos sobre ellos, lo cual sirve para probar nuestros
razonamientos.

En las próximas secciones vamos a trabajar sobre algunos elementos
constituyentes de la lógica, que nos permiten combinar elementos, y vamos a
entender mejor algunas utilidades prácticas de la lógica.

\section{Conectivas}
\label{chap:logica:sec:conectivas}

Desviémonos por un momento sobre la lógica como ciencia e imaginemos que en
nuestro barrio han montado una moderna verdulería, atendida pura y
exclusivamente por un robot. El verdulero robot es capaz de atender a los
clientes, responder sobre el stock de las frutas y verduras, y venderlas, claro.
Cómo es un robot, su repertorio de trabajo es límitado, y solo responde a cosas
muy puntuales, y realizar preguntas que están fuera del repertorio del robot
solo logra que este no nos responda.

Lo que nos va a interesar en esta analogía son las preguntas sobre el stock. La
versión inicial del robot solo comprende preguntas simples sobre una fruta o
verdura, para indicarnos si hay en stock, o no hay en stock. Así, solo podemos
realizar preguntas de la forma ``\textit{¿hay $X$?}'', donde $X$ es una fruta o
verdura, por ejemplo, ``\textit{¿hay sandías?}'' o ``\textit{¿hay cebolla?}''.

\wraprimage[0.3]{unidades/3_logica/1_logica/imagenes/sweeper.jpg}
{Sweeper, un robot que recolecta pimientos morrones.} {Fotografía de
http://www.sweeper-robot.eu}

Ahora bien, un cierto día queremos preparar una ensalada de frutas, un popular
postre que, bien preparado, debe llevar al menos tres ingredientes, manzana,
banana y naranja. No disponemos de los elementos para la preparación en casa,
por lo que nos aventuramos a la verdulería, a preguntar al robot si tiene stock
de los ingredientes que necesitamos.

Por supuesto, no podemos preguntar al verdulero robot si tiene o no los
ingredientes de una ensalada de frutas, pues nada sabe sobre cuestiones
culinarias. Es decir, si bien nosotros queremos preguntarle al robot
``\textit{¿hay ingredientes para una ensalada de frutas?}'', no podemos hacerlo,
y debemos recurrir a preguntas más simples, basadas en lo que el robot entiende.
Por tanto debemos realizar tres preguntas independientes: ``\textit{¿hay
manzanas?}'', ``\textit{¿hay bananas?}'' y ``\textit{¿hay naranjas?}''.

El robot, por supuesto, podría responder con ``\textit{si}'' o ``\textit{no}'' a
cualquiera de esas preguntas, y es nuestro trabajo deducir cuándo se puede
preparar la ensalada de frutas y cuándo no.

Pensemos en un posible escenario. El robot respondió con ``\textit{si}'' a
``\textit{¿hay manzanas?}'', con ``\textit{si}'' a ``\textit{¿hay bananas?}'' y
con ``\textit{no}'' a ``\textit{¿hay naranjas?}''. En este caso, \textbf{no}
podremos preparar la ensalada de frutas, pues nos faltaría un ingrediente, las
naranjas. Podemos apreciar que la única respuesta viable a cada pregunta, para
que podamos elaborar nuestro postre, debe ser ``\textit{si}''.

Si pudieramos interactuar con un robot más avanzado, que entienda preguntas un
poco más complejas, tal vez preguntaríamos algo del estilo ``\textit{¿hay
manzanas, bananas y naranjas?}'', es decir, uniríamos las tres preguntas en una
sola, usando una palabra especial para ``\textbf{conectar}'' esos elementos, el
``\textit{y}'' (gramaticalmente, la coma entre manzanas y bananas representa lo
mismo que un ``\textit{y}'').

Pensemos otro ejemplo antes de ir a una definición. Si queremos preparar un
bizcochuelo, debemos agregar la ralladura de un cítrico noble, como las naranjas
o los limones. Nuevamente, no podemos ir al verdulero robot y preguntarle si
tiene los ingredientes para un bizcochuelo, solo podemos preguntar por frutas
específicas, ``\textit{¿hay limones?}'' y ``\textit{¿hay naranjas?}''.

Analicemos una vez más las respuestas posibles. Si el verdulero nos responde con
``\textit{si}'' a ambas preguntas, podremos elegir que cítrico preferímos. Si
responde a una con ``\textit{si}'' y la otra con ``\textit{no}'' tendremos que
usar el cítrico disponible (ya sea el limón o las naranjas), pero aún podremos
preparar el bizcochuelo. Solo si ambas respuestas son ``\textit{no}'' es que se
nos hará imposible preparar el mismo.

En una versión más avanzada del robot preguntaríamos algo como ``\textit{¿hay
limones o naranjas?}''. Notar que acá la palabra que usamos para conectar los
elementos es ``\textit{o}'', y no ``\textit{y}''. Estas palabras para conectar
son lo que llamamos \textbf{conectivas}, y tienen una connotación semántica
asociada. No es lo mismo preguntar ``\textit{¿hay manzanas y bananas?}'' que
preguntar ``\textit{¿hay manzanas o bananas?}''. Podemos pensar estas preguntas
como compuestas de partes más pequeñas, y la respuesta a alla depende de las
respuestas individuales a cada una de las preguntas que la componen, así como de
si usamos ``\textit{y}'' o si usamos ``\textit{o}'' para unirlas.

La siguiente tabla muestra en cada fila un escenario de posibles respuestas,
donde el robot contesta a cada una de las preguntas individuales únicamente, y
cómo interpretaríamos la pregunta compuesta según la conectiva elegida.

\vspace{0.5cm}
\centerline{
    \begin{tabular}{ c | c | c | c }
          \textbf{¿hay manzanas?}
        & \textbf{¿hay bananas?} & \textbf{¿hay manzanas y bananas?} &
        \textbf{¿hay manzanas o bananas?} \\
        \hline
        Si & Si & Si & Si \\
        Si & No & No & Si \\
        No & Si & No & Si \\
        No & No & No & No \\
    \end{tabular}
}

Así, la \textbf{conectiva} utilzada es importante, porque cambia la semántica de
la pregunta. Esto es algo que probablemente nos resulte obvio de nuestra vida
cotidiana, donde usamos las palabras ``\textit{y}'' y ``\textit{o}'' de forma
habitual, y esperamos una connotación semántica identica (o al menos similar) a
la arriba mencionada.

Con esta idea en mente estamos listos para dar una definición sobre las
conectivas en términos de la lógica.

\begin{definition}
    Una \textbf{conectiva} es un elemento que permite unir una o dos entidades
    portadoras de valor de verdad, para conformar una entidad más grande, cuyo
    valor de verdad depende de las entidades constituyentes y de la semántica
    asociada a la conectiva.
\end{definition}

En estos sencillos ejemplos, podríamos pensar a nuestras preguntas como las
unidades portadoras de verdad, aunque veremos en los próximos capítulos a qué
nos referímos con esto cuando veamos lógicas con sistemas formales específicos.
Entonces, dicho de otra forma, y en nuestro ejemplo del verdulero robot, una
conectiva permite unir una o dos preguntas para conformar una pregunta más
grande.

Pensemos por un momento en las siguientes preguntas:
\begin{itemize}
    \item ``\textit{¿hay manzanas y bananas?}''
    \item ``\textit{¿hay manzanas y también bananas?}''
    \item ``\textit{¿hay manzanas y además bananas?}''
\end{itemize}

Se puede apreciar como todas las formas de preguntar refieren a lo mismo. Para
que la pregunta completa se responda como ``\textit{si}'', esperaríamos que las
respuestas individuales de ``\textit{¿hay manzanas?}'' y ``\textit{¿hay
bananas?}'' sean ``\textit{si}''. Los \textbf{lenguajes
naturales}\index{Lenguaje natural}, como el español o el inglés, suelen ser muy
ricos y diversos, e incluyen sinónimos y diversas construcciones gramaticales
que son capaces de expresar la misma idea de distintas formas.

A partir de ahora entonces, cuando tengamos una conectiva, no vamos a optar por
realizar una única pregunta con la conectiva en el medio (ej. ``\textit{¿hay
manzanas y bananas?}'') sino que en su lugar vamos a preferir mencionar las
preguntas que componen a ese elemento como preguntas individuales, de forma de
dejar más claro la conectiva entre ellas (ej. ``\textit{¿hay manzanas? Y ¿hay
bananas?}''). En la siguiente sección veremos que la lógica opta por utilizar
símbolos específicos para cada conectiva, lo que evita las ambiguedades causadas
por elementos gramaticales.

Pero ¿Por qué decimos que pueden unir una entidad sola? ¿Con qué se une? Bueno
se une con la connotación semántica de la conectiva en sí misma. Volvamos a la
verdulería robot y veamos un nuevo ejemplo.

Sofía es una nena muy glotona que disfruta mucho de comer la ensalada de frutas
que prepara su mamá. Pero su mamá no puede preparar ensalada de frutas todos los
días, ya que no siempre están disponibles los ingredientes en la verdulería.
Sofía acaba de cumplir años, y comienza a sospechar de su madre, pensando que
tal vez le dice que no hay ingredientes para preparar ensalada de frutas, cuando
en realidad es solo que no tiene ganas de cocinar. Sin embargo, como ahora ya
tiene edad suficiente para salir a la calle sola, decide ir a la verdulería por
su cuenta para corroborar que su madre, quien acaba de decirle que no puede
preparar ensalada de frutas porque no hay manzanas, está diciendo la verdad.
Sofía se acerca a la verdulería y su primer intento de interacción con el robot
verdulero consiste en preguntarle ``\textit{¿Es cierto que no hay manzanas?}'',
respuesta que el robot responde sorpresivamente para los presentes con
``\textit{si}''.

Se pregunta entonces al lector, ¿Estaba mintiendo la mamá de Sofía? Pensemos
primero en la pregunta de Sofía y la respuesta del verdulero: ``\textit{¿Es
cierto que no hay manzanas?}'', con la consiguiente respuesta ``\textit{si}''.
Ese ``\textit{sí}'' no refiere a ``\textit{¿hay manzanas?}'', sino a si es
cierto lo que Sofía está diciendo a continuación de ``\textit{es cierto qué}'',
lo cual es ``\textit{no hay manzanas}''. Si la respuesta del verdulero fue
``\textit{si}'' quiere decir que es cierto, por lo que efectivamente
``\textit{no hay manzanas}'', mientras que si hubiera sido ``\textit{no}'',
sería porque no es cierto que ``\textit{no hay manzanas}'', es decir, que habría
manzanas.

Si bien parece un trabalenguas, lo que está expresando la pregunta de Sofía es
una pregunta que está relacionada a ``\textit{¿hay manzanas?}'' pero que es
inversa a esta. Es decir, si la respuesta a ``\textit{¿hay manzanas?}'' era
``\textit{si}'', la respuesta a la pregunta de Sofía debería ser
``\textit{no}'', mientras que si la respuesta a ``\textit{¿hay manzanas?}'' es
``\textit{no}'', la respuesta a la pregunta de sofía debería ser
``\textit{si}''.

En este ejemplo también hay una conectiva, la idea de cambiar la respuesta, o lo
que en general llamamos ``\textit{no}''. Cuando trabajamos con preguntas en
español, a veces uno tiende a preguntar cosas como ``\textit{¿no hay
mayonesa?}'', esperando que nos respondan ``\textit{si}'' y nos pasen la
mayonesa en el caso de que hubiera (por ``\textit{si hay}'' y no por
``\textit{si, es cierto que no hay}''). En español, así como en muchos otros
idiomas, es raro usar la idea semántica de ese ``\textit{no}'' en preguntas.

Sin embargo, si pensamos en afirmaciones, decir ``\textit{hay mayonesa}'' y
decir ``\textit{no hay mayonesa}'' son claramente cosas distintas, siendo una
afirmación opuesta a la otra, algo sobre lo que ahondaremos en los siguientes
capítulos. Este es un ejemplo de una conectiva que trabaja sobre una única
entidad portadora de valor de verdad, cambiando su valor. Veremos más sobre las
distintas conectivas en la siguiente sección.

El concepto de \textbf{conectiva} es transversal a todas las formas de la
lógica, ya que modela un elemento fundamental del lenguaje natural, que tiene
una connotación semántica importante en las oraciones que componemos.

Cómo mencionamos al inicio de esta unidad, la gracia de la lógica consiste en
poder analizar la forma de los razonamientos y sus elementos constituyentes, por
lo que no nos interesa cual es la (o las) palabra exacta que se está utilizando
como conectiva, sino cuál es la semántica que se le está dando a esa palabra.
Esto permite identificar la conectiva, y por tanto entender mejor el rol de los
elementos involucrados. Por eso nos interesa ``\textbf{la forma}'', y no el
contenido.

En esta sección vamos a profundizar y formalizar las conectivas que hemos
trabajado en los ejemplos en la \autoref{chap:logica:sec:conectivas}. Si bien
existen otras conectivas, abordaremos las mismas recién en el
\autoref{chap:logica_proposicional}.

Como parte de esta formalización, dejaremos atrás la idea de ``\textit{si}'' y
de ``\textit{no}'' como respuestas a nuestras preguntas, para pasar a hablar en
lenguaje lógico, teniendo entonces ``\textit{verdadero}'' ($\ltruefull$, o
simplemente $\ltrue$) y ``\textit{falso}'' ($\lfalsefull$, o simplemente
$\lfalse$) en su lugar.

\subsection{Conjunción}
\index{Conjuncion@Conjunción}
\label{chap:logica:subsec:and}

La \textbf{conjunción} es la conectiva que permite unir dos elementos formando
uno más grande, cuyo valor de verdad es verdadero solo cuando \textbf{ambos
elementos son también verdaderos}, y falso cuando cualquiera de los elementos
que lo constituyen es falso. Hemos visto esta conectiva en el ejemplo de la
ensalada de frutas, en expresiones como ``\textit{¿hay manzanas? Y ¿hay
bananas?}''.

\wraprimage[0.3]{unidades/3_logica/1_logica/imagenes/venn_logica_1.png}
{Diagramas de Venn de las operaciones lógicas.} {Elaboración propia.}

La conjunción se asocia en el español a la palabra ``\textit{y}'', pero también
a otra serie de palabras o frases como: ``\textit{también}'',
``\textit{además}'', ``\textit{adicionalmente}'', ``\textit{en adición}'',
``\textit{más}'', ``\textit{incluso}'', ``\textit{inclusive}'',  entre otras
varias \footnote{ Puede verse una lista más amplia en el
    \autoref{anex:logica}
}.

Ya vimos formas de expresar la misma pregunta a nivel semántico con elementos
gramaticales o sintácticos diferentes en la sección anterior, y mencionamos
también que a cada conectiva se le asocia un símbolo distinto. El símbolo para
la conjunción es ``$\land$''. Así, podríamos expresar la pregunta del ejemplo
como ``\textit{¿hay manzanas? $\land$ ¿hay bananas?}''.

Para entender mejor la conectiva podemos pensar en cualquier par de elementos
portadores de valor de verdad genéricos, ``$\Phi$'' (phi) y ``$\Psi$'' (psi),
que son unidos por la conectiva de conjunción, quedando ``$\Phi \land \Psi$''.
Así, podemos analizar el valor semántico de dicha conjunción de forma genérica,
analizando el valor de verdad de cada una de las partes que lo constituyen.

\begin{itemize}
    \item Si $\Phi$ es $\ltruefull$ y $\Psi$ es $\ltruefull$, entonces $\Phi
        \land \Psi$ es $\ltruefull$
    \item Si $\Phi$ es $\ltruefull$ y $\Psi$ es $\lfalsefull$, entonces $\Phi
        \land \Psi$ es $\lfalsefull$
    \item Si $\Phi$ es $\lfalsefull$ y $\Psi$ es $\ltruefull$, entonces $\Phi
        \land \Psi$ es $\lfalsefull$
    \item Si $\Phi$ es $\lfalsefull$ y $\Psi$ es $\lfalsefull$, entonces $\Phi
        \land \Psi$ es $\lfalsefull$
\end{itemize}

Esto quiere decir que, si reemplazamos $\Phi$ y $\Psi$ por preguntas cualquiera,
y las unimos con una conjunción, la semántica asociada será la arriba expuesta,
independientemente de la pregunta que usemos.

Podemos expresar más cómodamente cada uno de esos casos posibles y los
resultados de la conjunción en una tabla, en donde hay una columna para cada uno
de los elementos constituyentes de la conjunción ($\Phi$ y $\Psi$) y una columna
para el resultado de la conjunción ($\Phi \land \Psi$). En cada fila se analiza
una situación con valores de verdad distintos para los elementos constituyentes.
Obtenemos entonces la siguiente tabla:

\centerline{
    \begin{tabular}{ c | c | c }
        \textbf{$\Phi$} & \textbf{$\Psi$} & \textbf{$\Phi \land \Psi$}\\
        \hline
        $\ltruefull$  & $\ltruefull$  & $\ltruefull$  \\
        $\ltruefull$  & $\lfalsefull$ & $\lfalsefull$ \\
        $\lfalsefull$ & $\ltruefull$  & $\lfalsefull$ \\
        $\lfalsefull$ & $\ltruefull$  & $\lfalsefull$ \\
    \end{tabular}
}

Llamamos a esta tabla la \textbf{tabla semántica de la conectiva}\index{Tabla
semantica@Tabla semántica}, ya que muestra la semántica asociada a la misma.
Cada conectiva tendrá su propia tabla semántica.

\subsection{Disyunción}
\index{Disyuncion@Disyunción}
\label{chap:logica:subsec:or}

Mientras que la conjunción requiere que ambos elementos sean verdaderos, la
\textbf{disyunción} solo pide que \textbf{al menos alguno lo sea}. La disyunción
es la idea del ``\textit{o uno, u otro, o ambos}''.

En general asociamos la disyunción con la palabra ``\textit{o}'', como en
ejemplo del bizcochuelo que trabajamos con anterioridad, ``\textit{¿hay limones?
O ¿hay naranjas?}''. Sin embargo, acá también pueden haber otras palabras o
frases del español que pueden ser identificadas como conectivas de disyunción,
como ``\textit{o también}'', ``\textit{o bien}'', ``\textit{aunque de pronto}''
o construcciones como ``\textit{tal vez..., tal vez...}'' o ``\textit{puede... o
puede...}''\footnote{ Nuevamente se puede consultar el \autoref{anex:logica}
para una lista más amplia. }

En este caso el símbolo que se utiliza es $\lor$ \footnote{ Notar que es el
    mismo símbolo de la conjunción, pero invertido, lo cual a veces trae
    confusión. Se recomienda mantener referencia de la simbología a mano hasta
    memorizarla. Puede ver la simbología en el \autoref{anex:logica}. }

La tabla semántica para la conectiva, con entidades genéricas, quedaría de la
siguiente forma:

\centerline{
    \begin{tabular}{ c | c | c }
        \textbf{$\Phi$} & \textbf{$\Psi$} & \textbf{$\Phi \lor \Psi$}\\
        \hline
        $\ltruefull$  & $\ltruefull$  & $\ltruefull$  \\
        $\ltruefull$  & $\lfalsefull$ & $\ltruefull$  \\
        $\lfalsefull$ & $\ltruefull$  & $\ltruefull$  \\
        $\lfalsefull$ & $\ltruefull$  & $\lfalsefull$ \\
    \end{tabular}
}

\subsection{Negación}
\index{Negacion@Negación}
\label{chap:logica:subsec:not}

La conjunción y la disyunción son conectivas \textbf{binarias}\index{Conectiva
binaria}, es decir, unen dos elementos. Por su lado, la \textbf{negación} es una
conectiva \textbf{unaria}\index{Conectiva unaria}, es decir que trabaja sobre un
único elemento.

La \textbf{negación} es la idea de ``\textit{no}'', en el lenguaje natural,
aunque se pueden usar otras frases como ``\textit{no es cierto que}'',
``\textit{es falso que}'', entre otras.

El símbolo es $\lnot$ y su semántica es la de \textbf{cambiar el valor de verdad
de la entidad sobre la que opera}. Ya vimos el ejemplo de la pregunta de Sofía
al verdulero robot, la que podemos expresar entonces como ``\textit{$\lnot$ ¿hay
manzanas?}''.

La tabla de su semántica general es la siguiente:

\centerline{
    \begin{tabular}{ c | c }
        \textbf{$\Phi$} & \textbf{$\lnot \Phi$}\\
        \hline
        $\ltruefull$  & $\lfalsefull$ \\
        $\lfalsefull$ & $\ltruefull$  \\
    \end{tabular}
}

Estas tres conectivas son las más simples que podemos encontrar, y las
utilizamos de forma cotidiana en el lenguaje natural, pero como ya mencionamos,
no son las únicas. Sin embargo, debemos primero acostumbrarnos a trabajar con
conectivas y valores de verdad antes de poder ver conectivas más complejas.

\section{Utilidad de la lógica en la informática}
\label{chap:logica:sec:utilidad}

Cabe preguntarse después de entender el objeto de estudio y sus elementos básicos, cuál es el punto de estudiar esta ciencia en las disciplinas informáticas en general.


\section{La lógica como formuladora de preguntas}
\label{chap:logica:subsec:formuladora_preguntas}

Una caracteristica interesante que podemos apreciar entonces es la capacidad que
nos brindan las conectivas para pensar preguntas complejas en términos de otras
preguntas más simples. Podemos decir por ejemplo que la pregunta ``\textit{¿hay
para preparar ensalada de fruta?}'' es lo mismo que preguntar ``\textit{¿hay
manzanas? $\land$ ¿hay bananas? $\land$ ¿hay naranjas?}''. Lo mismo vale para la
pregunta ``\textit{¿hay para preparar bizcochuelo?}'' que queda expresada como
``\textit{¿hay limones? $\lor$ ¿hay naranjas?}''.

Las computadoras, al igual que nuestro verdulero robot, tienen un repertorio
limitado en cuánto a lo que son capaces de hacer y responder. De hecho, como ya
vimos, solo hacen cuentas, y una ``respuesta'' de la computadora, no es más que
electricidad (o ausencia de ella) en un cable especifico. Incluso aplicando el
concepto de caja negra, y subiendo varios niveles en la abstracción de las
ciencias de la computación, las respuestas posibles de una computadora son
limitadas.

Así, esta caracteristica de la lógica es de especial relevancia en las ciencias
de la computación, en diversas área, aunque vamos a centrarnos en este caso en
el área de la programación.

Pensemos en una caracteristica que tienen hoy muchos teléfonos celulares
notebooks o computadoras de escritorio, cambiar el esquema de colores del equipo
cuando se hace de noche, el llamado ``modo oscuro'', que se activa de forma
automática.

Para poder realizar un programa que active automaticamente el modo oscuro, un
programador debería contar con que la computadora es capaz de responder una
simple pregunta, ``\textit{¿es de noche?}''. Pero las computadoras en general,
no incluyen circuitería específica para poder responder una pregunta tan
abstracta. Pensemos además, que esa pregunta, se responde no solo diferente en
diferentes horas del día, sino en diferentes días y regiones, por lo que no es
posible realizar una solución en hardware para algo tan complejo.

De esta forma, un programador debe encontrar algún conjunto de preguntas que la
computadora sepa responder, y que le sirvan para responder lo que verdaderamente
le interesa, saber si ``\textit{¿es de noche?}''. Bien, las computadoras suelen
incorporar un reloj, por lo que pueden decir la hora del día (asumiendo que la
hora esté configurada adecuadamente). Además hoy en día es normal contar con
conexión a internet, y los servicios de sistemas meteorológicos nacionales
suelen publicar en diversos lugares la hora del ocaso. Si puedieramos obtener
ambos datos, saber si es o no de noche implica simplemente comparar la hora del
ocaso con la hora del equipo( si el equipo tiene una hora posterios, es de
noche, sino, aún es de día).

Este ejemplo es tal vez un poco dificil de entender, pues además de preguntas
que se responden con verdadero o falso hay también otros elementos involucrados,
como fechas y comparaciones, pero es un ejemplo real que busca simplemente que
el lector se haga una idea de cómo los programadores deben buscar ``reformular''
sus preguntas en términos de otros elementos disponibles en el equipo.

En la Unidad \ref{unit:programacion}, dedicada a programación, veremos como esta
capacidad de reformular preguntas se vuelve sumamente útil para resolver algunos
problemas computacionales.

\section{Precedencia de conectivas}
\label{chap:logica:sec:precedencia}

Vimos cómo se pueden unir dos preguntas con conectivas de conjunción y
disyunción, y como se le puede cambiar el valor de verdad mediante la negación.
También vimos tablas para entender la semántica de dichas conectivas en casos
genéricos, y cómo podemos tener preguntas más complejas como ``\textit{¿hay
manzanas? $\land$ ¿hay bananas? $\land$ ¿hay naranjas?}'', que involucran más de
dos preguntas pequeñas.

Vamos a aventurarnos a continuación en comprender como \textbf{operan} las
conectivas lógicas en preguntas complejas, donde hay múltiples conectivas de
distinto tipo, para, ya en la \autoref{chap:logica:sec:tablas_de_verdad} poder
entender la forma de operar con estos elementos. Para esto, vamos a llamar
\textbf{elementos base} a cada pregunta símple, es decir, las preguntas que no
tienen conectivas (Del estilo de las que el verdulero robot sabría responder) y
\textbf{expresión} a una pregunta compleja, formada por uno o varios elementos
base unidos con conectivas.

Y es que en primer lugar hay que entender que, para el analisis de la lógica en
términos formales, las conectivas lógicas operan de forma similar a las
operaciones matemáticas. Por ejemplo, la suma es también una operación binaria
entre elementos, al igual que la conjunción, con la diferencia de que esos
elementos son números, en lugar de elementos portadores de valor de verdad.

Veamos un caso concreto para entender como operar en situaciones con múltiples
conectivas. En matemáticas estamos acostumbrados a trabajar con múltiples
operaciones en una misma expresión, y tenemos convenciones para trabajar con las
mismas para determinar de que forma llegar al resultado. Por ejemplo, podemos
tener una cuenta expresada en la forma ``$((2+4) \times ((7+3)^2))$''. Si bien
la expresion tiene muchos elementos y operadores entre ellos, los
\textbf{paréntesis} ayudan a desambiguar el orden en el que debemos resolver
cada operación. Sin embargo, no solemos escribir absolutamente todos los
paréntesis. Por ejemplo, los más externos, nunca los escribimos, y tampoco el de
la potenciación. Más aún, pensemos en la expresión ``$7 \times 5 + 4 \times 6$''
y pensemos en la forma de resolverla. La multiplicación asocia más fuerte que la
suma, debiendo resolverse primero (lo que llamamos coloquialmente ``separar en
términos''), por lo que incluso sin paréntesis involucrados, existe un orden en
el que las operaciones deben realizarse. Por último, en el caso de una expresión
como ``$2 + 3 + 4$'', existe un orden implicito de operación, de izquierda a
derecha, por lo que en realidad es lo mismo que expresar ``$(2 + 3) + 4$'', y la
forma de operar involucra solucionar primero ``$(2 + 3)$'', para luego, al
resultado, sumarle ``$4$''.

En lógica ocurren cosas muy similares. Si queremos entender el valor de verdad
de una pregunta con varias partes, como ``\textit{¿hay manzanas? $\land$ ¿hay
bananas? $\land$ ¿hay naranjas?}'', tenemos que comprender cómo se asocian esos
elementos base, para interpretar las conectivas en el orden correcto, y obtener
un valor de verdad al final de las mismas que sea adecuado. El orden en el que
realizamos las operaciones se conoce como \textbf{precedencia}.

\begin{definition}\index{precedencia} La \textbf{precedencia} es el orden en el
    que se deben resolver los diversos operadores de una fórmula determinada,
    debiendo respetarla para una resolución adecuada.
\end{definition}

Por un lado, y al igual que en matemáticas, al momento de formalizar, se pueden
utilizar \textbf{paréntesis} para indicar de forma explicita la precedencia.
Así, decir ``\textit{(¿hay manzanas? $\land$ ¿hay naranajas?) $\lor$ ¿hay
limones?}'' no es lo mismo que decir ``\textit{¿hay manzanas? $\land$ (¿hay
naranajas? $\lor$  ¿hay limones?)}''. En cualquier caso, lo que está dentro de
los paréntesis tiene precedencia por lo que está fuera, por lo que primero
debería resolverse lo que está dentro. La semántica de cada una de esas
expresiones es por tanto distinta\footnote{ Notar que no existe una propiedad
asociativa entre conectivas distintas, aunque si existen para algunas
conectivas. Veremos propiedades de algunas conectivas en el
\autoref{chap:logica_proposicional}. }.

Podemos usar paréntesis para desambiguar la semántica de una expresión que
involucra múltiples conectivas, pero la gran cantidad de paréntesis puede
volverse confusa rápidamente. Por ello existen operaciones que tienen
precedencia sobre otra (también se dice que tienen prioridad o más jerarquía).
Esto es algo a lo que también estamos acostumbrados de las matemáticas, donde la
expresión $2 + 3 \times 5$ debe ser resuelta primero por la multiplicación, y
luego la suma, ya que la multiplicación tiene mayor precedencia.

En la lógica la precedencia es la siguiente:
\begin{enumerate}
    \item Negación
    \item Conjunción y Disyunción
\end{enumerate}

Es decir, si tenemos una expresión como ``\textit{¿hay manzanas? $\lor$ ($\lnot$
¿hay naranajas? $\land$ ¿hay limones?)}'', ya que no hay paréntesis, la forma en
que debe interpretarse es: primero la negación, que como es un operador unario,
asociará un único elemento; luego como los elementos tienen igual precedencia,
leeremos de izquierda a derecha, salvo que haya paréntesis, como es este caso.
Entonces, la interpretación correcta es  ``\textit{¿hay manzanas? $\lor$
(($\lnot$ ¿hay naranajas?) $\land$ ¿hay limones?)}'' y no ``\textit{¿hay
manzanas? $\lor$ ($\lnot$ (¿hay naranajas? $\land$ ¿hay limones?))}''.

El orden de precedencia cobrará especial importancia cuando contemos con un
sistema formal subyacente más elaborado y empecemos a formalizar todos los
elementos que componen a nuestras expresiones, así como cuando deseemos realizar
pruebas, comprender la validez de los razonamientos, etc. Por ahora, esta
explicación debería ser suficiente para lo que nos convoca en este capítulo. Una
versión completa con el resto de las conectivas existentes se presentará en el
\autoref{anex:logica}.

\section{Analisis de expresiones}
\index{Analisis de expresiones}
\label{chap:logica:sec:analisis_expresiones}

En la sección anterior vimos que podemos tener múltiples preguntas unidas con
varias conectivas, como en el caso de ``\textit{¿hay manzanas? $\lor$ $\lnot$
¿hay naranajas? $\land$ ¿hay limones?}''. También, que las operaciones de las
conectivas son unarias o binarias, por lo que hay un orden en el que deben
realizarse las mismas al momento de interpretar las expresiones completas de
forma semántica.

Pero como hacemos entonces poder saber el valor de verdad que tiene una
expresión compuesta de varias preguntas y conectivas. Es decir, ante una serie
de respuestas concretas a las preguntas que componen una expresión, por ejemplo,
si sabemos que la respuesta a ``\textit{¿hay manzanas?}'' es ``$\ltruefull$'', a
``\textit{¿hay naranajas?}'' es ``$\lfalsefull$'' y a ``\textit{¿hay limones?}''
es ``$\lfalsefull$'', ¿Cuál es el valor de verdad de la expresión? ¿Cómo llego a
dicho valor? Lo que necesitamos es una forma de``calcular'' dicho valor.

En primer lugar debemos comprender otro concepto clave de la lógica, la idea de
\textbf{valuación}, de la que veremos primero su definición.

\begin{definition}\index{Valuación@Valuación} Una \textbf{valuación} es la
    asignación de un valor de verdad concreto a cada elemento base portador de
    verdad que participa en una expresión concreta.
\end{definition}

Un ejemplo de valuación sería decir que ``\textit{¿hay manzanas?}'' es
``$\ltruefull$'', ``\textit{¿hay naranajas?} '' es ``$\lfalsefull$'' y
``\textit{¿hay limones?}'' es ``$\lfalsefull$''. Por otro lado, si
``\textit{¿hay manzanas?}'' es ``$\lfalsefull$'', ``\textit{¿hay naranajas?}''
es ``$\ltruefull$'' y ``\textit{¿hay limones?}'' es ``$\lfalsefull$'' entonces
tenemos otra valuación, diferente a la primera.

Es decir, una valuación consiste en asignar valores de verdad concretos a cada
una de los elementos que componen la expresión a evaluar. La cantidad de
valuaciones posibles de una expresión está dada por la cantidad de elementos
base en la misma, es decir, la cantidad de preguntas elementales a responder.

Si por ejemplo, hay una única pregunta, entonces tendremos solamente dos
valuaciones posibles (o bien le asignamos verdadero o bien falso a dicha
pregunta). Si tenemos dos preguntas, pasaremos a tener cuatro valuaciones
posibles, y si tenemos tres preguntas, serán ocho. Como regla general podemos
decir entonces que:

\begin{corollary}
    $\text{cantidad de valuaciones diferentes} = 2^\text{cantidad de elementos
    base}$
\end{corollary}

Así, si deseamos conocer el valor de la expresión para una valuación concreta,
se puede realizar un analisis semántico mediante resolución operacional, paso a
paso para cada conectiva. Sin embargo, no es la única forma de analizar una
expresión compleja, ya que en ocaciones queremos poder resolver la expresión
para todas las valuaciones posibles al mismo tiempo, para lo cual necesitaremos
una herramienta llamada tabla de verdad.

\subsection{Análisis mediante semántica}
\index{Análisis mediante semántica}
\label{chap:logica:subsec:analisis_semantico}

Si sabemos los valores concretos de los elementos base, es decir, \textbf{si
contamos con una valuación}, podemos analizar la expresión completa
``\textbf{resolviendo}'' cada una de las conectivas. Para ello, nos valdremos de
las tablas semánticas de cada una de las conectivas y de la descomposición de la
expresión en partes.

Veamos el ejemplo de la expresión antes mencionada, ``\textit{¿hay manzanas?
$\lor$ $\lnot$ ¿hay naranajas? $\land$ ¿hay limones?}''. El primer paso es
determinar la precedencia, para saber el orden en el que debemos evaluar los
distintos elementos. Ya hemos realizado este paso en la sección anterior, y
determinamos la precedencia como ``\textit{(¿hay manzanas? $\lor$ (($\lnot$ ¿hay
naranajas?) $\land$ ¿hay limones?))}''. Con esto en mano, podemos resolver desde
los paréntesis más internos a los más externos, reemplazando para cada pregunta
por el valor de verdad correspondiente para la valuación que se está analizando.
En este caso iremos por la valuación donde ``\textit{¿hay manzanas?}'' es
``$\ltruefull$'', ``\textit{¿hay naranajas?} '' es ``$\lfalsefull$'' y
``\textit{¿hay limones?}'' es ``$\lfalsefull$''.

\begin{example}
    \sindent Valuación a analizar:

    \dindent ``\textit{¿hay manzanas?}'' es ``$\ltruefull$''

    \dindent ``\textit{¿hay naranajas?} '' es ``$\lfalsefull$''

    \dindent ``\textit{¿hay limones?}'' es ``$\lfalsefull$''

    ~

    \sindent Expresión a analizar:

    \dindent ``\textit{¿hay manzanas? $\lor$ $\lnot$ ¿hay naranajas? $\land$
    ¿hay limones?}''

    ~

    \sindent Expresión con precedencia analizada:

    \dindent ``\textit{(¿hay manzanas? $\lor$ (($\lnot$ ¿hay naranajas?) $\land$
    ¿hay limones?))}''

    ~

    \sindent Resolución paso a paso:

    \dindent ``\textit{(¿hay manzanas? $\lor$ (($\lnot$ ¿hay naranajas?) $\land$
    ¿hay limones?))}''

    \dindent \qquad {\scriptsize $\rightarrow$ Se reemplaza \textit{¿hay
    naranjas?} por el valor que tiene en la valuación, $\lfalsefull$.}

    \dindent ``\textit{(¿hay manzanas? $\lor$ (($\lnot$ $\lfalsefull$) $\land$
    ¿hay limones?))}''

    \dindent \qquad {\scriptsize $\rightarrow$ Se resuelve el $\lnot$ según su
    tabla semántica, para $\lfalsefull$ la tabla indica que la negación debe ser
    $\ltruefull$.}

    \dindent ``\textit{(¿hay manzanas? $\lor$ ($\ltruefull$ $\land$ ¿hay
    limones?))}''

    \dindent \qquad {\scriptsize $\rightarrow$ Se reemplaza \textit{¿hay
    limones?} por el valor que tiene en la valuación, $\lfalsefull$.}

    \dindent ``\textit{(¿hay manzanas? $\lor$ ($\ltruefull$ $\land$
    $\lfalsefull$))}''

    \dindent \qquad {\scriptsize $\rightarrow$ Se resuelve el $\land$ según su
    tabla semántica, para el caso donde el primer elemento es $\ltruefull$ y el
    segundo}

    \dindent \qquad {\scriptsize es $\lfalsefull$ la tabla indica que la
    conjunción debe ser $\lfalsefull$.}

    \dindent ``\textit{(¿hay manzanas? $\lor$ $\lfalsefull$)}''

    \dindent \qquad {\scriptsize $\rightarrow$ Se reemplaza \textit{¿hay
    manzana?} por el valor que tiene en la valuación, $\ltruefull$.}

    \dindent ``\textit{($\ltruefull$ $\lor$ $\lfalsefull$)}''

    \dindent \qquad {\scriptsize $\rightarrow$ Se resuelve el $\lor$ según su
    tabla semántica, para el caso donde el primer elemento es $\ltruefull$ y el
    segundo}

    \dindent \qquad {\scriptsize es $\lfalsefull$ la tabla indica que la
    disyunción debe ser $\ltruefull$.}

    \dindent ``\textit{$\ltruefull$}''
\end{example}

Como se puede apreciar, la forma de resolver es muy similar a lo que uno
realizaría en matemáticas, con la salvedad de que aquí no hay números, sino
valores de verdad. Conocer las operaciones, y el resultado que se obtiene de
cada una ante ciertos valores de verdad, es importante para poder resolver cada
paso, y se puede recurrir a las tablas semánticas de cada conectiva para ello.

Sin embargo, esta metodología tiene la limitación de solo permitir determinar el
valor de verdad final para una única valuación. Si se desea saber el resultado
para otra valuación, se deberá volver a realizar todos los pasos, reemplazando
ahora cada pregunta por los valores de la nueva valuación.

Recordemos que la cantidad de valuaciones es de dos, elevado a la cantidad de
elementos base de la expresión. Para expresiones con dos elementos base, ya
serán cuatro veces las que deberemos realizar el análisis, algo poco viable.

Sin embargo, es ideal para cuando queremos determinar el valor de verdad de la
expresión sabiendo exactamente los valores de los elementos que la componen. Por
ejemplo, si alguien me respondió las preguntas.

Si lo que se desea es poder analizar todas las posibles valuaciones al mismo
tiempo, se debe recurrir a otra herramienta, las tablas de verdad.

\subsection{Tablas de verdad}
\label{chap:logica:subsec:tablas_de_verdad}

Una \textbf{tabla de verdad} permite visualizar y analizar todas las posibles
valuaciones, y como estas influyen en la expresión final y en cada una de sus
partes. Consiste en una tabla de doble entrada, en donde cada columna representa
un elemento portador de valor de verdad (un elemento base o una expresión
compuesta por conectivas) y cada fila una posible valuación.

La tabla de verdad es ideal para poder analizar expresiones donde hay varios
elementos base, pues permite una más rápida resolución para cada valuación que
la resolución paso a paso, pero también es útil para casos en donde no sabemos
cuál es la valuación que nos interesa. Por ejemplo, puede que nos interese saber
cuales son las valuaciones que hacen la expresión verdadera, y cuales no.

Aún así, necesitaremos tantas filas como valuaciones posibles, lo cual hace que
si tenemos muchas preguntas, las tablas de verdad dejen de ser una gran
herramienta para el análisis, pues ante 5 preguntas ya habrá que analizar 32
valuaciones diferentes, es decir, terminaremos con una tabla de 32 filas.
Existen otras formas de análisis que no veremos en esta sección.

Para ver cómo armar la tabla de verdad, resolveremos nuevamente el ejemplo de la
expresión ``\textit{¿hay manzanas? $\lor$ $\lnot$ ¿hay naranajas? $\land$ ¿hay
limones?}''. Ya sabemos que esta expresión contiene tres elementos base, y serán
por tanto ocho las valuaciones posibles ($2^3$). Para comenzar a armar nuestra
tabla deberemos arrancar entonces por tres columnas, una para cada pregunta
elemental, y 8 filas, que de momento estarán vacías.

\centerline{
\begin{tabular}{ c | c | c }
    \textit{¿hay manzanas?} & \textit{¿hay naranjas?} & \textit{¿hay limones?}
    \\
    \hline
    && \\
    && \\
    && \\
    && \\
    && \\
    && \\
    && \\
    &&
\end{tabular}
}

Ahora hay que completar esta parte de la tabla y asegurarnos de que contamos con
todas las valuaciones posibles, así como que no estamos duplicando ninguna
valuación. Para ello podemos completarlo de forma mecánica siguiendo los pasos
enunciados a continuación: Comenzando por la primer columna, completamos la
mitad de las filas con $\ltruefull$ y la otra mitad con $\lfalsefull$, quedando
la tabla como la siguiente.

\centerline{
\begin{tabular}{ c | c | c }
    \textit{¿hay manzanas?} & \textit{¿hay naranjas?} & \textit{¿hay limones?}
    \\
    \hline
    $\ltrue$  && \\
    $\ltrue$  && \\
    $\ltrue$  && \\
    $\ltrue$  && \\
    $\lfalse$ && \\
    $\lfalse$ && \\
    $\lfalse$ && \\
    $\lfalse$ &&
\end{tabular}
}

Pasamos a la segunda columna y hacemos lo mismo, pero esta vez completamos un
cuarto de los casos con $\ltruefull$, otro cuarto con $\lfalsefull$, y luego
repetimos un cuarto más con $\ltruefull$ y otro cuarto con $\lfalsefull$,
quedando así.

\centerline{
\begin{tabular}{ c | c | c }
    \textit{¿hay manzanas?} & \textit{¿hay naranjas?} & \textit{¿hay limones?}
    \\
    \hline
    $\ltrue$  & $\ltrue$  & \\
    $\ltrue$  & $\ltrue$  & \\
    $\ltrue$  & $\lfalse$ & \\
    $\ltrue$  & $\lfalse$ & \\
    $\lfalse$ & $\ltrue$  & \\
    $\lfalse$ & $\ltrue$  & \\
    $\lfalse$ & $\lfalse$ & \\
    $\lfalse$ & $\lfalse$ &
\end{tabular}
}

Volvemos a repetir el proceso, pero ahora tomando de a un octavo, con el primer
octavo como $\ltruefull$, el segundo como $\lfalsefull$ y repitiendo. La tabla
quedará así.

\centerline{
\begin{tabular}{ c | c | c }
    \textit{¿hay manzanas?} & \textit{¿hay naranjas?} & \textit{¿hay limones?}
    \\
    \hline
    $\ltrue$  & $\ltrue$  & $\ltrue$  \\
    $\ltrue$  & $\ltrue$  & $\lfalse$ \\
    $\ltrue$  & $\lfalse$ & $\ltrue$  \\
    $\ltrue$  & $\lfalse$ & $\lfalse$ \\
    $\lfalse$ & $\ltrue$  & $\ltrue$  \\
    $\lfalse$ & $\ltrue$  & $\lfalse$ \\
    $\lfalse$ & $\lfalse$ & $\ltrue$  \\
    $\lfalse$ & $\lfalse$ & $\lfalse$
\end{tabular}
}

De esta forma obtenemos todas las posibles valuaciones (notar como no hay dos
filas que se repitan en la tabla). Esta dinámica puede aplicarse
independientemente de la tabla y la cantidad de filas que tenga. Por ejemplo, si
hubiera solo dos preguntas, habrá solo 4 filas y 2 columnas, la primera se
completará tomando mitad y mitad, y la segunda tomando de a cuartos, para
completar toda la tabla. Si fueran cuatro preguntas, tendríamos 16 filas, y
habría un cuarto paso, tomando de a dieciseisavos.

Ahora estamos listos para resolver las conectivas de la tabla. El paso siguiente
es ir agregando columnas, una para cada operación, en el orden en que deben
resolverse según la precedencia. Ya hemos analizado la precedencia de la
expresión, y sabemos que la primer operación a resolver en este ejemplo es
``\textit{$\lnot$ ¿hay naranajas?}''. Por esto agregamos dicha columna a la
tabla.

\centerline{\small
\begin{tabular}{ c | c | c | c }
    \textit{¿hay manzanas?} & \textit{¿hay naranjas?} & \textit{¿hay limones?} &
    \textit{$\lnot$ ¿hay naranajas?}\\
    \hline
    $\ltrue$  & $\ltrue$  & $\ltrue$  & \\
    $\ltrue$  & $\ltrue$  & $\lfalse$ & \\
    $\ltrue$  & $\lfalse$ & $\ltrue$  & \\
    $\ltrue$  & $\lfalse$ & $\lfalse$ & \\
    $\lfalse$ & $\ltrue$  & $\ltrue$  & \\
    $\lfalse$ & $\ltrue$  & $\lfalse$ & \\
    $\lfalse$ & $\lfalse$ & $\ltrue$  & \\
    $\lfalse$ & $\lfalse$ & $\lfalse$
\end{tabular}
}

Para completar la columna ``\textit{$\lnot$ ¿hay naranajas?}'' debemos mirar la
o las preguntas involucradas en dicha operación. En este caso, la única parte
que aparece es ``\textit{¿hay naranajas?}'', es decir, la que figura en la
segunda columna de nuestra tabla. Así, deberemos mirar dicha columna y aplicar
para cada fila la semántica de la conectiva de negación.

La negación nos indica en su tabla semántica genérica que, para cualquier cosa
que negemos, si el valor era verdadero, debe pasar a ser falso, y si era falso,
debe pasar a ser verdadero. Luego, para completar la columna fila a fila,
debemos ir mirando el valor de la segunda columna para dicha fila, y aplicando
el cambio indicado por la conectiva en su tabla semántica. Por ejemplo, en la
primer fila el valor en la segunda columna es $\ltruefull$ por lo que deberemos
completar en la cuarta columna con $\lfalsefull$. Si seguimos el proceso para
cada fila de la tabla tedremos\footnote{Marcamos en la tabla con color celeste
la columna que acabamos de completar, y en amarilla la columna que miramos para
poder hacerlo. El color solo aporta mayor claridad al paso que acabamos de
mencionar, pero no es significativo en el proceso de resolución.}.

\centerline{\small
\begin{tabular}{ c | c | c | c }
    \textit{¿hay manzanas?} & \textit{¿hay naranjas?} & \textit{¿hay limones?} &
    \textit{$\lnot$ ¿hay naranajas?}\\
    \hline
    $\ltrue$  & \cellcolor{yellow}$\ltrue$  & $\ltrue$  &
    \cellcolor{cyan}$\lfalse$ \\
    $\ltrue$  & \cellcolor{yellow}$\ltrue$  & $\lfalse$ &
    \cellcolor{cyan}$\lfalse$ \\
    $\ltrue$  & \cellcolor{yellow}$\lfalse$ & $\ltrue$  &
    \cellcolor{cyan}$\ltrue$  \\
    $\ltrue$  & \cellcolor{yellow}$\lfalse$ & $\lfalse$ &
    \cellcolor{cyan}$\ltrue$  \\
    $\lfalse$ & \cellcolor{yellow}$\ltrue$  & $\ltrue$  &
    \cellcolor{cyan}$\lfalse$ \\
    $\lfalse$ & \cellcolor{yellow}$\ltrue$  & $\lfalse$ &
    \cellcolor{cyan}$\lfalse$ \\
    $\lfalse$ & \cellcolor{yellow}$\lfalse$ & $\ltrue$  &
    \cellcolor{cyan}$\ltrue$  \\
    $\lfalse$ & \cellcolor{yellow}$\lfalse$ & $\lfalse$ &
    \cellcolor{cyan}$\ltrue$
\end{tabular}
}

Continuamos con la operación ``\textit{$\lnot$ ¿hay naranajas? $\land$ ¿hay
limones?}''. En este caso, las columnas involucradas son la cuarta y la tercera.
Es decir, el resultado de ``\textit{$\lnot$ ¿hay naranajas?}'' se usa para
calcular ``\textit{$\lnot$ ¿hay naranajas? $\land$ ¿hay limones?}'. Nuevamente
entonces agregamos una columna, y seguimos la semántica, en este caso, de la
conjunción, que nos dice que para cada fila, debemos completar con verdadero
solo cuando ambas columnas involucradas son verdaderas, y con falso en cualquier
otro caso. Si seguimos esto obtendremos:

\centerline{\scriptsize
\begin{tabular}{ c | c | c | c | c }
    \textit{¿hay manzanas?} & \textit{¿hay naranjas?} & \textit{¿hay limones?} &
    \textit{$\lnot$ ¿hay naranajas?} & \textit{$\lnot$ ¿hay naranajas? $\land$
    ¿hay limones?}\\
    \hline
    $\ltrue$  & $\ltrue$  & \cellcolor{yellow}$\ltrue$ &
    \cellcolor{yellow}$\lfalse$ & \cellcolor{cyan}$\lfalse$ \\
    $\ltrue$  & $\ltrue$  & \cellcolor{yellow}$\lfalse$ &
    \cellcolor{yellow}$\lfalse$ & \cellcolor{cyan}$\lfalse$ \\
    $\ltrue$  & $\lfalse$ & \cellcolor{yellow}$\ltrue$  &
    \cellcolor{yellow}$\ltrue$  & \cellcolor{cyan}$\ltrue$ \\
    $\ltrue$  & $\lfalse$ & \cellcolor{yellow}$\lfalse$ &
    \cellcolor{yellow}$\ltrue$  & \cellcolor{cyan}$\lfalse$ \\
    $\lfalse$ & $\ltrue$  & \cellcolor{yellow}$\ltrue$  &
    \cellcolor{yellow}$\lfalse$ & \cellcolor{cyan}$\lfalse$ \\
    $\lfalse$ & $\ltrue$  & \cellcolor{yellow}$\lfalse$ &
    \cellcolor{yellow}$\lfalse$ & \cellcolor{cyan}$\lfalse$ \\
    $\lfalse$ & $\lfalse$ & \cellcolor{yellow}$\ltrue$  &
    \cellcolor{yellow}$\ltrue$  & \cellcolor{cyan}$\ltrue$ \\
    $\lfalse$ & $\lfalse$ & \cellcolor{yellow}$\lfalse$ &
    \cellcolor{yellow}$\ltrue$  & \cellcolor{cyan}$\lfalse$
\end{tabular}
}

La última operación de nuestra expresión es ``\textit{¿hay manzanas? $\lor$
$\lnot$ ¿hay naranajas? $\land$ ¿hay limones?}'', es decir, la expresión
completa, e implica resolver la disyunción entre la primera columna, y la quinta
columna, que acabamos de completar. Semánticamente la disyunción nos dice en su
tabla que debemos completar con verdadero cualquier fila que, en alguna de las
columnas involucradas, tenga por valor verdadero, y falso solamente cuando en
ambos casos es falso.

\centerline{\scriptsize
\begin{tabular}{ c | c | c | c | c | c }
    \tiny \textit{¿hay manzanas?} & \tiny \textit{¿hay naranjas?} & \tiny
    \textit{¿hay limones?} & \tiny \textit{$\lnot$ ¿hay naranajas?} & \tiny
    \textit{¿hay manzanas? $\land$ $\lnot$ ¿hay naranajas?} & \tiny \textit{¿hay
    manzanas? $\lor$ $\lnot$ ¿hay naranajas? $\land$ ¿hay limones?}\\
    \hline
    \cellcolor{yellow}$\ltrue$  & $\ltrue$  & $\ltrue$  & $\lfalse$ &
    \cellcolor{yellow}$\lfalse$ & \cellcolor{cyan}$\ltrue$  \\
    \cellcolor{yellow}$\ltrue$  & $\ltrue$  & $\lfalse$ & $\lfalse$ &
    \cellcolor{yellow}$\lfalse$ & \cellcolor{cyan}$\ltrue$  \\
    \cellcolor{yellow}$\ltrue$  & $\lfalse$ & $\ltrue$  & $\ltrue$  &
    \cellcolor{yellow}$\ltrue$  & \cellcolor{cyan}$\ltrue$  \\
    \cellcolor{yellow}$\ltrue$  & $\lfalse$ & $\lfalse$ & $\ltrue$  &
    \cellcolor{yellow}$\lfalse$ & \cellcolor{cyan}$\ltrue$  \\
    \cellcolor{yellow}$\lfalse$ & $\ltrue$  & $\ltrue$  & $\lfalse$ &
    \cellcolor{yellow}$\lfalse$ & \cellcolor{cyan}$\lfalse$ \\
    \cellcolor{yellow}$\lfalse$ & $\ltrue$  & $\lfalse$ & $\lfalse$ &
    \cellcolor{yellow}$\lfalse$ & \cellcolor{cyan}$\lfalse$ \\
    \cellcolor{yellow}$\lfalse$ & $\lfalse$ & $\ltrue$  & $\ltrue$  &
    \cellcolor{yellow}$\ltrue$  & \cellcolor{cyan}$\ltrue$  \\
    \cellcolor{yellow}$\lfalse$ & $\lfalse$ & $\lfalse$ & $\ltrue$  &
    \cellcolor{yellow}$\lfalse$ & \cellcolor{cyan}$\lfalse$
\end{tabular}
}

La última columna nos permite visualizar en que casos (en que valuaciones) la
expresión completa es verdadera, y en que casos no lo es. De hecho, salvo las columnas
para los elementos base, y la última columna que muestra la expresión completa, el resto de las columnas no son en general parte de interés, pues solo se realizan como pasos intermedios para llegar al valor de la expresión. Es decir, no podemos prescindir de elaborar esas columnas, porque sin ellas no podríamos arribar al resultado, pero la realidad es que, una vez completa la tabla, podemos eliminar esas columnas intermedias. Así, la tabla anterior quedaría como:

\centerline{
\begin{tabular}{ c | c | c | c }
    \textit{¿hay manzanas?} & \textit{¿hay naranjas?} &
    \textit{¿hay limones?} & \textit{¿hay
    manzanas? $\lor$ $\lnot$ ¿hay naranajas? $\land$ ¿hay limones?}\\
    \hline
    $\ltrue$  & $\ltrue$  & $\ltrue$  & $\ltrue$  \\
    $\ltrue$  & $\ltrue$  & $\lfalse$ & $\ltrue$  \\
    $\ltrue$  & $\lfalse$ & $\ltrue$  & $\ltrue$  \\
    $\ltrue$  & $\lfalse$ & $\lfalse$ & $\ltrue$  \\
    $\lfalse$ & $\ltrue$  & $\ltrue$  & $\lfalse$ \\
    $\lfalse$ & $\ltrue$  & $\lfalse$ & $\lfalse$ \\
    $\lfalse$ & $\lfalse$ & $\ltrue$  & $\ltrue$  \\
    $\lfalse$ & $\lfalse$ & $\lfalse$ & $\lfalse$
\end{tabular}
}

En esta sección nos
concentraremos únicamente en la estructura y en la resolución mecánica de una
tabla de verdad, más no procederemos a su análisis exhaustivo, concepto que
abordaremos en el \autoref{chap:logica_proposicional}.

Sin embargo, mencionaremos tres escenarios posibles que pueden darse tras
completar una tabla de verdad y algunas de sus implicancias.

Si en algunas valuaciones, la expresión final resulta $\ltruefull$ y en algunas
resulta $\lfalsefull$, decimos que estamos frente a una
\textbf{contingencia}\index{Contingencia}.

También puede darse el caso en el que no haya ninguna valuación que haga
verdadera la expresión, es decir, que para toda valuación, el resultado de la
expresión es $\lfalsefull$. Ante esta situación decimos que se trata de una
\textbf{contradicción}\index{Contradiccion@Contradicción}.

El caso contrario sería que no haya casos falsos, es decir, que para toda
valuación, la expresión sea $\ltruefull$. Esto se conoce con el nombre de
\textbf{tautología}\index{Tautologia@Tautología}.

Las tautologías y las contradicciones son interesantes de analizar, ya que son
lo que permite formular leyes lógicas. Veremos en el próximo capítulo como usar
este conocimiento para tal fin.

\section{Resumen}
\label{chap:logica:sec:resumen}

En este capítulo intentamos abordar la lógica desde una perspectiva intuitiva al
lector, comprendiendo no solo la definición e historia de esta ciencia, así como
su objeto de estudio, sino algunos de los diversos elementos transversales a la
disciplina. Sin embargo, no hemos establecido ningún sistema formal particular.

Incluso así, la cantidad de conceptos involucrados en esta introducción han sido
varios, por lo que intentaremos proporcionar un breve resumen al lector de lo
expuesto.

La \textbf{lógica} es una ciencia formal que estudia la \textbf{deducción} en
los \textbf{razonamientos}, es decir, los procesos de pensamiento del cerebro
humano que llevan a elaborar \textbf{conclusiones} a partir de una o más
\textbf{premisas}. Por ser una ciencia formal, a la lógica no le interesarán los
elementos que constituyen el razonamiento en sí, sino la \textbf{forma} que
tienen dichos razonamientos.

La lógica basa su análisis de \textbf{validez} de un razonamiento mediante la
operatoria con \textbf{valores de verdad} (binarios y dicotómicos,
\textbf{verdadero} y \textbf{falso}). En todo sistema formal de la lógica habrá
elementos que son portadores de valor de verdad (encaramos la idea pensando en
preguntas que se responden por ``\textit{si}'' o ``\textit{no}''.

Los distintos elementos portadores de valor de verdad pueden ser elementos base,
es decir estar solos, o formar parte de expresiones más complejas, mediante el
uso de \textbf{conectivas}. Una \textbf{conectiva} es un elemento fundamental de
la lógica, que permite unir elementos portadores de verdad, para elaborar un
nuevo elemento, más complejo, y cuyo valor semántico depende tanto de la
conectiva como de los elementos constituyentes. Puden ser \textbf{unarias} (como
la \textbf{negación}) o \textbf{binarias} (como la \textbf{conjunción} y la
\textbf{disyunción}).

Cuando en una expresión hay más de una conectiva, se pueden utilizar paréntesis
para desambiguar la \textbf{precedencia}. Además hay un orden de precedencia
implicito para cuando no se colocan paréntesis (negación, conjunción y luego
disyunción). La precedencia permite entender sin ambiguedad el significado
semántico de una expresión con múltiples conectivas, indicando qué elemento debe
ser ``resuelto'' primero, y cuál después.

Para analizar una expresión compleja podemos elaborar una \textbf{tabla de
verdad}, que nos permite visualizar facilmente todos los elementos involucrados,
así como cada posible \textbf{valuación}. La tabla de verdad se puede completar
de forma mecánica, aunque su interpretación y análisis son más complejos, por lo
que se dejarán para el próximo capítulo.

Según el resultado de la expresión en todas las valuaciones podemos clasificarlo
como: una \textbf{tautología} (todas verdaderas), una \textbf{contradicción}
(todas falsas) o una \textbf{contingencia} (algunas verdaderas, algunas falsas).

Si se tiene una valuación concreta, o se desea analizar solo un caso particular,
puede realizarse en su lugar una resolución semántica paso a paso.

Además vimos una de las utilidades de la lógica en las ciencias de la
computación, la capacidad de permitir \textbf{elaborar preguntas a partir de
otras más simples}.

En el \autoref{chap:logica_proposicional} veremos una primera definición formal
de la lógica, basada en los principios de Aristóteles, y profundizaremos sobre
varios de los elementos aquí trabajados, definiendo claramente quienes son los
elementos portadores de valor de verdad, ahondando en nuevas conectivas,
realizando el pasaje del lenguaje natural al lenguaje formal y viceversa, y
estableciendo formas de determinar si un razonamiento es o no válido mediante el
análisis de las tablas de verdad.

Luego veremos los límites de la lógica proposicional con los que se encontraron
los matemáticos de la Edad Moderna. Esto dará lugar a, ya en el
\autoref{chap:logica_predicados}, a trabajar la lógica de predicados, que
resuelve algunos de los problemas de la lógica proposicional, y permite
formalizar razonamientos que no eran posibles de formalizar con el sistema
anterior. Veremos allí sobre la forma de mencionar elementos, sus propiedades y
relaciones, y maneras de hablar sobre todos los elementos del universo, o sobre
algún elemento en particular, así como hacer aseveraciones sobre ellos.

\section{Actividades}
\label{chap:logica:sec:actividades}

\begin{exercise}
    Dados los siguientes razonamientos, indique si se tratan de razonamientos
    deductivos o inductivos.

    \begin{enumerate}[a)]
      \item
        Dado que las versiones más nuevas de los navegadores soportan HTML5,
        y dado que el Microsoft Edge es la última versión del navegador de dicha
        empresa, se sigue que el Microsoft Edge soporta HTML5.

      \item
        Mi primo compró una computadora aquí y tenía instalado Windows 10.
        Mi hermano compró una computadora aquí y tenía instalado Windows 10.
        Mi tío compró una computadora aquí y tenía instalado Windows 10.
        Si compro una computadora aquí tendrá instalado Windows 10.

      \item
        Maria se inscribió el jueves y solo tenía habilitadas las materias del CI.
        Pablo se inscribió el jueves y solo tenía habilitadas las materias del CI.
        Juan se inscribió el jueves y solo tenía habilitadas las materias del CI.
        Por lo tanto, todo alumno que se inscriba el jueves tiene solo habilitadas las materias del CI.
    \end{enumerate}
\end{exercise}

\begin{exercise}
    Considerando las siguientes preguntas como básicas:

    \begin{minipage}{0.45\textwidth}
        \begin{itemize}
            \item ¿hay harina?
            \item ¿hay manteca?
            \item ¿hay aceite?
            \item ¿hay agua?
        \end{itemize}
    \end{minipage}
    \begin{minipage}{0.45\textwidth}
        \begin{itemize}
            \item ¿hay huevos?
            \item ¿hay yerba?
            \item ¿hay chocolate?
            \item ¿hay azúcar?
        \end{itemize}
    \end{minipage}

    Se le pide que, utilizando las mencionadas como preguntas básicas\footnote{
        Recuerde que puede elaborar preguntas auxiliares que le ayuden a
        solucionar las pedidas } y las conectivas lógicas vistas, exprese las
        preguntas generales a continuación\footnote{ Se espera escriba algo de
        la forma ``\textit{¿hay manzanas?} $\land$ \textit{¿hay bananas?}'' }:
    \begin{enumerate}[a)]
        \item \textbf{¿hay para hacer una torta?} (Una torta requiere harina,
        huevos y manteca)
        \item \textbf{¿hay para hacer huevos fritos?} (Requiere huevos y aceite)
        \item \textbf{¿hay para hacer huevos duros?} (Requiere huevos y agua)
        \item \textbf{¿Puedo almorzar huevos?} (Ya sean duros o fritos)
        \item \textbf{¿hay para hacer una torta de chocolate?} (Idéntico a una
        torta, más chocolate)
        \item \textbf{¿Solo se puede tomar mate amargo?} (Cuando se puede tomar
        mate, es decir, hay agua y yerba, pero no hay azúcar)
        \item \textbf{¿No hay nada para el mate?} (Cuando se puede tomar mate,
        pero no hay torta de ningún tipo)
    \end{enumerate}
\end{exercise}

\begin{exercise}
    Considerando las siguientes preguntas básicas y sus respuestas asociadas:
    \begin{itemize}
        \item ¿Plutón es un planeta? $\lfalsefull$
        \item ¿Marte es un planeta?  $\ltruefull$
        \item ¿Marte es un satélite? $\lfalsefull$
        \item ¿Deimos es un satélite?  $\ltruefull$
        \item ¿Ganímedes es un satélite?  $\ltruefull$
        \item ¿Eros es un planeta? $\lfalsefull$
        \item ¿Eros es un satélite? $\lfalsefull$
    \end{itemize}

    Se le pide que, utilizando las mencionadas como preguntas básicas y las
    conectivas lógicas vistas, exprese las preguntas generales a continuación y
    determine el valor de verdad de cada una.
    \begin{enumerate}[a)]
        \item \textbf{¿Son Marte y Plutón planetas?}
        \item \textbf{¿Es Marte un planeta o es Plutón un planeta?}
        \item \textbf{¿Es Marte un planeta o un satélite?}
        \item \textbf{¿Es cierto que Marte es un planeta y Plutón no lo es?}
        \item \textbf{¿Es cierto que Ganímedes y Eros son satélites?}
        \item \textbf{¿Es cierto que Eros no es un satélite, pero Deimos si lo
        es?}
        \item \textbf{¿Es Deimos un satélito o es cierto que Marte es un
        planeta?}
        \item \textbf{¿Es Eros un satélite o es cierto que es un planeta?}
    \end{enumerate}
\end{exercise}

\begin{exercise}
    Considerando las siguientes preguntas como básicas:

    \begin{minipage}{0.45\textwidth}
        \begin{itemize}
            \item ¿hay ejercito enemigo al Norte?
            \item ¿hay ejercito enemigo al Este?
            \item ¿hay ejercito enemigo al Sur?
            \item ¿hay ejercito enemigo al Oeste?
        \end{itemize}
    \end{minipage}
    \begin{minipage}{0.45\textwidth}
        \begin{itemize}
            \item ¿hay ejercito aliado al Norte?
            \item ¿hay ejercito aliado al Este?
            \item ¿hay ejercito aliado al Sur?
            \item ¿hay ejercito aliado al Oeste?
        \end{itemize}
    \end{minipage}

    Se le pide que, utilizando las mencionadas como preguntas básicas\footnote{
        Recuerde que puede elaborar preguntas auxiliares que le ayuden a
        solucionar las pedidas } y las conectivas lógicas vistas, exprese las
        preguntas generales a continuación\footnote{ Se espera escriba algo de
        la forma ``\textit{¿hay manzanas?} $\land$ \textit{¿hay bananas?}'' }:
    \begin{enumerate}[a)]
        \item \textbf{¿Se está amenazado?} (Cuando hay ejercito enemigo en
        alguna dirección)
        \item \textbf{¿Se está libre de peligro?} (Cuando no hay ejercitos
        enemigos en ninguna dirección)
        \item \textbf{¿Se tiene apoyo?} (Cuando hay un ejercito aliado en alguna
        dirección)
        \item \textbf{¿Se está hasta las manos?} (Cuando no hay apoyo y se está
        amenazado)
        \item \textbf{¿Se puede neutralizar alguna amenaza?} (Cuando hay un
        ejercito enemigo en alguna dirección, pero también hay un ejercito
        aliado allí)
        \item \textbf{¿Se puede neutralizar todas las amenazas?} (Cuando se
        puede neutralizar en todas las direcciones)
    \end{enumerate}
\end{exercise}

\begin{exercise}
    Considerando las siguientes preguntas símples:
    \begin{itemize}
        \item ¿Se cumple P?
        \item ¿Se cumple Q?
        \item ¿Se cumple R?
    \end{itemize}

    Se le pide que analice que valuaciones darán verdadero y cuales falso para
        las siguientes preguntas compuestas\footnote{ Si pensar en preguntas
        como ``\textit{¿Se cumple P?}'' le resulta confuso, piense en
        reemplazarlas por preguntas que le sean más familiares, como
        ``\textit{¿Se cumple que el lobo vive en el bosque?}''. Lo interesante
        del ejercicio es que la pregunta en cuestión no es realmente relevante,
        sino la forma de la pregunta compuesta. }. Además se le pide determine
        cuales de ellas se tratan de tautologías, cuales de contradicciones y
        cuáles de contingencias.
    \begin{enumerate}[a)]
        \item \textbf{¿Se cumple P? $\land$ $\lnot$ ¿Se cumple P?}
        \item \textbf{¿Se cumple P? $\lor$ $\lnot$ ¿Se cumple P?}
        \item \textbf{¿Se cumple P? $\land$ ¿Se cumple Q?}
        \item \textbf{¿Se cumple P? $\land$ (¿Se cumple Q? $\lor$ ¿Se cumple
        R?)}
        \item \textbf{¿Se cumple P? $\lor$ ¿Se cumple Q?}
        \item \textbf{$\lnot$ ¿Se cumple P? $\lor$ (¿Se cumple Q? $\land$ ¿Se
        cumple R?)}
        \item \textbf{$\lnot$ ¿Se cumple P? $\lor$ $\lnot$ (¿Se cumple Q? $\lor$
        ¿Se cumple R?)}
    \end{enumerate}
\end{exercise}

\begin{exercise}
    Sabiendo que las siguientes preguntas compuestas evalúan todas a
    $\ltruefull$
    \begin{itemize}
    \item ¿La palmera está creciendo torcida? y ¿Es cierto que el árbol no dio
    paltas este año?
    \item ¿Es cierto que no hay flores en el cantero? o ¿El árbol dio paltas
    este año?
    \end{itemize}

    Se pide que responda las siguientes preguntas simples
    \begin{enumerate}[a)]
        \item ¿El árbol dio paltas este año?
        \item ¿Hay flores en el cantero?
        \item ¿La palmera está creciendo torcida?
    \end{enumerate}
\end{exercise}

\begin{exercise}
    Dadas las siguientes preguntas base y sus respuestas:
    \begin{itemize}
        \item ¿Es Microsoft el creador de Windows? $\ltruefull$
        \item ¿Es Microsoft el creador de GNU? $\lfalsefull$
        \item ¿Es Richard Stallman el creador de Windows? $\lfalsefull$
        \item ¿Es Richard Stallman el creador de GNU? $\ltruefull$
    \end{itemize}

    Analice las preguntas a continuación, detecte las conectivas involucradas y
    reemplantee las preguntas para que estén elaboradas en términos de las
    preguntas dadas. Luego, analice el valor de verdad de las mismas.

    \begin{enumerate}[a)]
        \item ¿Es cierto que no es cierto que Microsoft es el creador de GNU?
        \item ¿Es cierto que Richard Stallman es el creador de Windows o de GNU?
        \item ¿Es cierto que Microsoft no es el creador de GNU pero que Richard
        Stallman si lo es?
        \item ¿Es cierto que GNU no es del mismo creador que Windows?
    \end{enumerate}
\end{exercise}

\begin{exercise}
    Se ha encontrado vida en otro planeta, y se ha decidido nombrar a los animales
    encontrados como ``Woofle'', ``Brlfks'' y ``Morlock''.

    Cada animal tiene sus características distintivas (pueden ser grandes o chicos,
    con o sin pelo, acuáticos o terrestres).

    Si las respuestas a todas las preguntas siguientes son $\ltruefull$, enumere que
    características tiene cada animal. Para ello, realice la tabla de verdad de
    cada pregunta y analice las respuestas a las preguntas base en las valuaciones
    verdaderas.

    \begin{itemize}
      \item ¿El Woofle tiene pelo y no es acuático?
      \item ¿El Morlock es terrestre y es grande, y el Brlfks es acuático?
      \item ¿El Brlfks es pequeño o el Morlock es pequeño?
      \item ¿El Woofle no es grande o el Brlfks es grande? ¿Y es cierto que el Morlock no es pequeño?
      \item ¿El Woofle tiene pelo y el Brlfks no? ¿O es cierto que el Brlfks no tiene pelo y el Morlock tiene pelo?
    \end{itemize}
\end{exercise}