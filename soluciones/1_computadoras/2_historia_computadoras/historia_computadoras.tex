\chapterimage{unidades/1_computadoras/2_historia_computadoras/imagenes/cover}
\chapterimagedescription{Museo de Historia de las Computadoras en Mountain View,
California, EE.UU.}
\chapterimageauthor{Michael Kappel}

\chapter{Historia de las computadoras}

\setcounter{section}{4}
\section{Actividades}
\index{Actividades}

\begin{exercise}
¿Cuál fue la primer computadora en argentina? ¿Para que se usaba? ¿Quién la
trajo al país?
\end{exercise}

La primer \textbf{computadora para fines científicos} que funcionó en Argentina
fue \textbf{Clementina}, una \textbf{Ferranti Mercury}, creada en 1957, de la
cual se produjeron solo 19 unidades.

\image{soluciones/1_computadoras/2_historia_computadoras/imagenes/manuel_sadosky_y_clementina.jpg}
{Manuel Sadosky en el Instituto de Cálculo trabajando con la computadora
Clementina junto a su colega Juan Carlos Angio.} {Fotografía de Archivo. Diario
Chaco}

\textbf{Manuel Sadosky} fue quien lideró las gestiones para su adquisición en
1959. La computadora llegó el 24 de Noviembre de 1960, aunque recién comenzó a
operar en Enero de 1961. Ferranti (una empresa inglesa) ganó la licitación por
sobre firmas como IMB, Remington y Philco. La máquina costo 152.099 libras
esterlinas (equivalentes a unos 4.500.000 dólares actuales, aproximadamente), lo
cual representó el desembolso más grande en ciencia y tecnología en la historia
hasta ese momento.

La computadora se ubicó en el \textbf{Instituto de Cálculo}, dependiente de la
\textbf{Universidad de Buenos Aires}, en el \textbf{Pabellón I de Ciudad
Universitaria, en Nuñez}.

El equipo tenía más de 4500 válvulas termiónicas (qué debían se reemplazadas en
varias oportunidades durante el funcionamiento de la máquina), y memoria de
núcleos magnéticos de 4 KWords (de 10 bits). Se constituía de 14 gabinetes de
60cm que tenían las funciones de procesador y memoria, 4 gabinetes con cilindros
magnéticos (para un total de 64 KWords de 10 bits), que ocupaban toda una
habitación. A esto había que sumarle otros 5 racks en otra habitación que
contenían las fuentes de alimentación. Medía en total una 50.000 veces más que
un gabinete de computadora moderna.

Carecía de monitor y de teclado. La entrada de instrucciones se hacía con un
lector fotoeléctrico de cinta de papel perforado, y los resultados se emitían
por una perforadora de cinta a 30 caracteres por segundo, opcionalmente
alimentando una teletipo a la velocidad standard de 7 caracteres por segundo.
Más adelante se le pudo adaptar un lector de tarjetas perforadas de fabricación
nacional, siendo este un método de ingreso de datos más práctico que el
original.

El lenguaje de programación que utilizaba era Mercury Autocode, especialmente
desarrollado para este modelo. Sobre Clementina se creó el primer lenguaje de
computación argentino, llamado \textbf{COMIC}. Fue creado por \textbf{Wilfred
Duran} y estaba adaptado a problemas de simulación socio económicos.

La computadora prestó servicios para varias dependencias del Estado, trabajando
en cálculos astronómicos (verificación de los cálculos manuales hechos por el
astrónomo ítalo-argentino Francisco J. Bobone sobre el pasaje del cometa Halley
en 1904), modelos matemáticos de cuencas fluviales y econométricos, desarrollo
en computadora del método de camino crítico (CPM), estudios de mecánica del
sólido, problemas lingüísticos y problemas estadísticos.

\image{soluciones/1_computadoras/2_historia_computadoras/imagenes/clementina.jpg}
{Imágen de la computadora Clementina.} {Fotografía de Archivo. Facultad de
Ciencias Económicas, UBA}

El nombre de Clementina surgió de una canción popular estadounidense \textbf{Oh
My Darling, Clementine} que venía entre los programas de muestra provistos por
Ferranti. La computadora tenía la posibilidad de accionar un parlante ubicado en
la consola, lo que permitía generar tonos muy rudimentarios por software. Luego,
utilizando dicho parlante se produjeron programas que tocarían tangos.

Clementina siguió funcionando hasta mediados del año 1971, cuando su
mantenimiento por falta de piezas se hizo imposible. Sería desmantelada y los
restos dispuestos para su eliminación como simples residuos. Tan sólo unos pocos
módulos fueron rescatados por personal técnico de la facultad antes de que se
los vendiera como chatarra, y aún los conservan como piezas de colección.
\vspace{1cm}

\begin{exercise}
¿Qué es la ley de Moore? ¿Se cumple actualmente?
\end{exercise}


La \textbf{ley de Moore} es una ley empírica postulada con el cofundador de
Intel \textbf{Gordon E. Moore} que expresa que \textbf{el número de transistores
de un microprocesador se duplica aproximadamente cada dos años}.

La ley fue postulada en 1965, por el joven ingeniero Gordon Moore era director
de los laboratorios de \textbf{Fairchild Semiconductor}. Él observó en los
primeros días de la microelectrónica, una tendencia que definía el mercado de
los semiconductores. Su observación anticipaba que la cantidad de circuitos
integrados se duplicaría cada año, con la reducción mensurable en costo. Poco
despues, en 1968, fundaría Intel junto con Robert Noyce.

En 1975, modificaría su propia ley al establecer que la duplicación no se
realizaría cada año, sino cada dos años.

\image{soluciones/1_computadoras/2_historia_computadoras/imagenes/gordon_moore.png}
{El científico y empresario Gordon E. Moore.} {Captura del video ``Cientificos
que debes conocer'', creado por la Chemical Heritage Foundation}

La consecuencia directa de la ley de Moore es que los precios de los
procesadores bajan al mismo tiempo que las prestaciones suben: la computadora
que hoy vale 3000 dólares costará la mitad al año siguiente y estará
``obsoleta'' en dos años.

También funciona como guía para las empresas que producen semiconductores, por
lo que es permite planificar a futuro que procesos deben realizar para que su
tecnología se mantenga relevante con respecto a la competencia.

\begin{quote}
    En Intel trabajamos duro para asegurarnos de que la ley de Moore continúe
    guiando a nuestra industria en el futuro. Ya hemos visualizado los próximos
    10 a 15 años de adelantos en nuestros laboratorios de investigación.
    \begin{flushright}
    Craig Barrett, CEO de Intel Corporation
    \end{flushright}
\end{quote}

La ley de Moore sigue vigente en la actualidad, a pesar de que hoy contamos con
dispositivos que utilizan procesadores de la más diversa índole, como notebooks,
tablets, teléfonos celulares, etc. Sin embargo, el mismo Moore ha dicho que su
ley quedará eventualmente obsoleta (en 10 o 15 años) cuando una nueva tecnología
reemplace a los semiconductores actuales.
\vspace{1cm}

\begin{exercise}
¿De quién son los cables submarinos de Internet?
\end{exercise}

Hay miles de cables submarinos que son la columna vertebral de Internet. Estos
cables, no son de acceso público y gratuito, sino se son privados, y una serie
de empresas son dueños de los mismos.

Cuando uno contrata un servicio de Internet hogareño, como \textbf{Speedy},
\textbf{Fibertel}, \textbf{Telecentro}, \textbf{Claro}, etc. o incluso 4G, quien
brinda el servicio (Conocido como \textbf{ISP}, por \textbf{Internet Service
Provider}), debe poder comunicarse a cualquier lugar del mundo (para permitirte
acceder a cualquier lado vía Internet). Para ello, debe utilizar los cables
submarinos, y por tanto, debe pagarle a su dueño.

A su vez, a quien le paga nuestro ISP puede ser dueño de algunos cables (Por
ejemplo, el que va de Argentina a Estados Unidos) pero no de otros (El que va de
Estados Unidos a Europa), y por tanto, puede que este también deba pagar a un
tercero para utilizar un cable.

De esta forma, se suele dividir a la columna vertebral de internet (llamada
generalmente \textbf{Internet Backbone}), en capas (\textbf{tiers}). Así,
nuestro ISP suele ser un considerado una empresa en el ``tier 3'', que utiliza
cables de una empresa en el ``tier 2'', o de una del ``tier 1''. Las del ``tier
2'' a su vez, utilizan los servicios de una empresa del ``tier 1''. Una empresa
se considera del ``tier 1'' cuando no debe pagar a nadie por el tráfico que
transmite por los cables.

Así, hay una serie de empresas que son de cierta forma, dueñas finales de los
cables que sostienen a internet. Algunas de ellas son \textbf{AT\&T},
\textbf{Verizon}, \textbf{CenturyLink}, \textbf{Sprint}, \textbf{Deutsche
Telekom AG}, \textbf{GTT Comunications}, \textbf{Orange}, \textbf{Telefónica},
\textbf{Tata Communications}, \textbf{Telia Carrier}, \textbf{Telecom Italia},
\textbf{NTT Communications}, \textbf{Liberty Global}, \textbf{KPN
International}, \textbf{PCCW Global} y \textbf{Zayo Group}.
\vspace{1cm}

\begin{exercise}
Sabía que en Argentina se desarrollaron varias distribuciones de Linux. Averigue
el nombre y la historia de algunas de ellas.
\end{exercise}


Argentina fue lugar de nacimiento de la primera distribución en ser reconocida
como totalmente libre por el Proyecto GNU, \textbf{Ututo Linux}, creada por
\textbf{Diego Saravia de la Universidad Nacional de Salta} en el año 2000. Su
nombre remite a una especie de lagartija típica de la región noroeste.
Lamentablemente, por falta de financiamiento y gente el proyecto dejo de
actualizarse en el 2013. Hoy en día hay pequeños proyectos que tienen la idea de
reflotar estra distribución.

\textbf{Tuquito Linux}, otra distribución argentina creada en Tucumán por
\textbf{Ignacio Díaz, Chris Arenas, y Mauro Torres}. Tuquito remite al nombre
con el que se conoce a un insecto de abdomen luminiscente en Tucumán, en general
llamado también luciérnaga. Tuquito intentó transformarse en una distribución de
escritorio nacional, sin éxito. Tras encontrarse software malicioso en los
servidores que alojaban a Tuquito, y la comunidad acusó a sus creadores de
alojarlo allí de forma intensional. Tras el abandono de sus usuarios el proyecto
murió en 2012.

\textbf{Lihuen} es una distribución Linux originalmente basada en GnuLinEx y
luego en Debian, desarrollada por la \textbf{Facultad de Informática de la
Universidad Nacional de La Plata}. El proyecto comenzó en 2004 con la intención
de realizar una distribución de Linux especialmente diseñada para la educación
en la universidad, incluyendo software pre-instalado y con la posibilidad de
funcionar en software medianamente antiguo. El proyecto va hoy en día en su
versión 6, y el LINTI, dependiente de la UNLP, es el encargado del mantenimiento
y desarrollo del sistema.

\textbf{Dragora} es otro sistema 100\% libre, recomendado por la FSF. El sistema
fue creado desde cero, resultando en un sistema similar a Slackware, aunque
incompatible en algunos aspectos con este.

Existen otros varios que han sido desarrollados por programadores argentinos en
conjunto con programadores de otros lados del mundo, como \textbf{Musix},
\textbf{Wandoo}, \textbf{RXart}, \textbf{FriceOS} o \textbf{Urli}.

\wraplimage{soluciones/1_computadoras/2_historia_computadoras/imagenes/huayra_vaca.jpg}
{La vaca voladora, mascota de Huayra Linux.} {Imágen de enREDando.}

Finalmente, la distribución Linux argentina más ampliamente extendida en los
últimos años ha sido \textbf{Huayra Linux}. Creada por \textbf{Educ.ar SE}, una
empresa estatal qué realizó el sistema en el marco del programa \textbf{Conectar
Igualdad}, viniendo el mismo integrado en las computadoras del programa. El
sistema operativo tiene por mascota una vaca voladora, haciendo referencia a un
chiste interno, en donde decían que las computadoras del programa contarían con
un sistema operativo propio ``el día que las vacas vuelen''. El sistema incluía
herramientas de administración para docentes, permitiendoles crear aulas
virtuales, un reproductor de televisión digital terrestre (TDA), y herramientas
para la enseñanza de programación, física, quimica, matemática, y otras áreas de
enseñanza. Tras cancelar el plan conectar igualdad en 2016, el proyecto comenzó
un proceso de desfinanciamiento por parte del Estado Nacional, quien era su
principal promotor. Actualmente ya no se planifican nuevas versiones, y la
situación del soporte a las versiones actuales varía entre escasa y nula.
\vspace{1cm}

\begin{exercise}
¿Conoce alguna de las microcomputadoras emblemática? ¿Cuál? Si no conoce
ninguna, investigue que computadoras marcaron esa época.
\end{exercise}

Cuando iniciaron las microcomputadoras eran pocas las empresas que se dedicaban
a fabricarlas. Hubieron ciertas marcas que vendieron un gran número de
ejemplares y que marcaron a todas las computadoras por venir, además de forjar
generaciones enteras de usuarios que aún hoy atesoran estas máquinas, las cuales
suelen tener altos valores en el mercado de colección.

Apple desarrolló en 1977 el \textbf{Apple II}, su primer computadora fabricada
en masa. Diseñada por \textbf{Steve Wozniak} el Apple II original marcaría una
era de éxito para Apple, y la arquitectura de base de la máquina sería utilizada
en diversas generaciones del equipo que se vendería hasta 1992 con modelos como
el Apple II Plus, Apple IIe, Apple IIc y el Apple IIGS. También aparecerían
cientos de clónicos de estos equipos. Apple descontinuaría los equipos tras
reemplazarlos con sus \textbf{Macintosh}, que comenzaron a producirse en 1984,
pero fueron un fracaso en ventan, logrando que la empresa despida a su fundador
original, \textbf{Steve Jobs}. Macintosh también tendría varias ediciones, como
Macintosh II, Macintosh Plus o Macintosh SE. Las ventas comenzarían a remontar
recién en 1990, y en 1994 el sistema evolución cambiando completamente la
arquitectura.

\image{soluciones/1_computadoras/2_historia_computadoras/imagenes/old_computers_museum.jpg}
{De izquierda a derecha, una TRS-80, una Commodore 64 y una Apple II.} {Imágen
del Museo de Historia de las Computadoras.}

También en 1977, Commodore, otra gran empresa de microcomputadoras de la época
lanzaría la \textbf{Commodore PET}, la primer computadora completamente equipada
de la compañia, y que marcaría el diseño de toda la gama de computadoras de 8
bits de la empresa, incluyendo \textbf{Commodore VIC-20}, su sucesora, la
\textbf{Commodore 64} y finalmente la \textbf{Commodore 128}. La Commodore 64
tenía esa denominación pues contaba solamente con 64KB de memoria RAM (hoy las
máquinas tienen en general 128000 veces más memoria). Commodore 64 se transformó
en un clásico, pasando a ser la principal competencia de la Apple II. Aunque se
seguirían vendiendo hasta 1994, las ventas comenzarían a disminuir, por lo que
en 1985 Commodore lanzaría al mercado la \textbf{Commodore Amiga}.

En 1977 también aparecía en los mercados la \textbf{TRS-80}, o Tandy Radio Shack
80. Tandy, empresa de tecnología dueña de una serie de tiendas de venta de
tecnología al menudeo, Radio Shack, decidió diseñar y vender su propio equipo.
El mismo se hizo sumamente popular por su escaso bajo costo. El equipo pasaría
por diversos modelos y se vendería hasta mediados de los 80.

En Reino Unido salió en 1982 la \textbf{ZX Spectrum}, por parte de
\textbf{Sinclair Research}. La computadora costaba solo 100 libras esterlinas
(equivalente al costo de un celular hoy en día). Esto la transformó en la
computadora más vendida del reino, dominando los mercados por los años
venideros.

Si bien hay muchisimas otras máquinas iconicas, podemos terminar por mencionar
la \textbf{IBM PC}, cuyo diseño de periféricos encastrables con ranuras
estandarizadas transformó la venta de hardware, abriendole la puerta a miles de
productores más pequeños que podían dedicarse a fabricar componentes
especializados en lugar de máquinas enteras.
\vspace{1cm}

\begin{exercise}
¿Qué lenguajes de programación conoce? ¿Sabe para que sirve y que
características tiene?
\end{exercise}

Hay cientos de lenguajes de programación, y constantemente aparecen nuevos
lenguajes. Los lenguajes pueden agruparse en diversas categorías, según
distintos criterios. Por ejemplo, podemos agruparlos en \textbf{lenguajes de
propósito general} (Cuando permiten programar cualquier cosa) o
\textbf{lenguajes de dominio específico} (Cuando sirven para programar algo en
particular, por ejemplo, videojuegos). También se los puede agrupar según que
tan bien ese lenguaje permite expresar las ideas del programador, y que tanto
puedo olvidarme de como se compone internamente una computadora al usar ese
lenguaje. De esta forma, tenemos \textbf{lenguajes de alto nivel} (Aquellos que
permiten expresar muy bien las ideas, y me permiten olvidarme de la máquina) y
de \textbf{bajo nivel} (Expresan mejor el funcionamiento interno de la máquina,
pero se vuelve más difícil expresar las ideas del programador). Otra posible
categorización es mediante paradigmas (Es decir, la forma general en la que se
estructura un programa en dicho lenguaje), donde tenemos \textbf{lenguajes
imperativos} (también llamados \textbf{estructurados}), \textbf{lenguajes
orientados a objetos}, \textbf{lenguajes funcionales}, y \textbf{lenguajes
lógicos}. Además hay lenguajes que entran en más de una de dichas categorías.

Entre los lenguajes más populares y conocidos se encuentran \textbf{C},
\textbf{C++}, \textbf{Java}, \textbf{C\#}, \textbf{JavaScript}, \textbf{Python},
\textbf{Ruby}, \textbf{Kotlin}, \textbf{Swift}, \textbf{Visual Basic},
\textbf{PHP}, \textbf{Pascal}, \textbf{Pearl}, \textbf{SQL}, \textbf{Go},
\textbf{Smalltalk}, \textbf{Haskell}, \textbf{COBOL}. La agencia TIOBE, genera
un listado con los lenguajes ``más populares''.


