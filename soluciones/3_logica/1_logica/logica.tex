\chapterimage{unidades/3_logica/1_logica/imagenes/cover}
\chapterimagedescription{Red Lógica de un Enrutador con LEDs}
\chapterimageauthor{Fotografía de tschoenemeyer}

\chapter{Introducción a la lógica}
\label{chap:logica}

\section{Actividades}
\label{chap:logica:sec:actividades}

\begin{exercise}
    Dados los siguientes razonamientos, indique si se tratan de razonamientos
    deductivos o inductivos.

    \begin{enumerate}[a)]
      \item
        Dado que las versiones más nuevas de los navegadores soportan HTML5, y
        dado que el Microsoft Edge es la última versión del navegador de dicha
        empresa, se sigue que el Microsoft Edge soporta HTML5.

      \item
        Mi primo compró una computadora aquí y tenía instalado Windows 10. Mi
        hermano compró una computadora aquí y tenía instalado Windows 10. Mi tío
        compró una computadora aquí y tenía instalado Windows 10. Si compro una
        computadora aquí tendrá instalado Windows 10.

      \item
        Maria se inscribió el jueves y solo tenía habilitadas las materias del
        CI. Pablo se inscribió el jueves y solo tenía habilitadas las materias
        del CI. Juan se inscribió el jueves y solo tenía habilitadas las
        materias del CI. Por lo tanto, todo alumno que se inscriba el jueves
        tiene solo habilitadas las materias del CI.
    \end{enumerate}
\end{exercise}

\begin{solution}
    Si bien basta con decir si es deductivo o inductivo, podemos además agregar qué
    tipo de razonamiento inductivo es, ya que podemos encontrar de dos tipos. En un
    razonamiento inductivo la conclusión puede ser un nuevo caso particular, que lo
    hace inductivo por analogía, o una generalización (es decir, afirmar que todos
    los casos futuros cumplirán una caracteristica), lo cual lo hace inductivo por
    enumeración.

    De esta forma, las respuestas son:

    \begin{enumerate}[a)]
        \item Razonamiento deductivo.
        \item Razonamiento inductivo (por analogía).
        \item Razonamiento inductivo (por enumeración).
    \end{enumerate}
\end{solution}

\begin{exercise}
    Considerando las siguientes preguntas como básicas:

    \begin{minipage}{0.45\textwidth}
        \begin{itemize}
            \item ¿hay harina?
            \item ¿hay manteca?
            \item ¿hay aceite?
            \item ¿hay agua?
        \end{itemize}
    \end{minipage}
    \begin{minipage}{0.45\textwidth}
        \begin{itemize}
            \item ¿hay huevos?
            \item ¿hay yerba?
            \item ¿hay chocolate?
            \item ¿hay azúcar?
        \end{itemize}
    \end{minipage}

    Se le pide que, utilizando las mencionadas como preguntas básicas\footnote{
        Recuerde que puede elaborar preguntas auxiliares que le ayuden a
        solucionar las pedidas } y las conectivas lógicas vistas, exprese las
        preguntas generales a continuación\footnote{ Se espera escriba algo de
        la forma ``\textit{¿hay manzanas?} $\land$ \textit{¿hay bananas?}'' }:
    \begin{enumerate}[a)]
        \item \textbf{¿hay para hacer una torta?} (Una torta requiere harina,
        huevos y manteca)
        \item \textbf{¿hay para hacer huevos fritos?} (Requiere huevos y aceite)
        \item \textbf{¿hay para hacer huevos duros?} (Requiere huevos y agua)
        \item \textbf{¿Puedo almorzar huevos?} (Ya sean duros o fritos)
        \item \textbf{¿hay para hacer una torta de chocolate?} (Idéntico a una
        torta, más chocolate)
        \item \textbf{¿Solo se puede tomar mate amargo?} (Cuando se puede tomar
        mate, es decir, hay agua y yerba, pero no hay azúcar)
        \item \textbf{¿No hay nada para el mate?} (Cuando se puede tomar mate,
        pero no hay torta de ningún tipo)
    \end{enumerate}
\end{exercise}

\begin{solution}
    Las respuestas son:

    \begin{enumerate}[a)]
        \item \textbf{\textit{¿Hay harina?} y \textit{¿Hay huevos?} y \textit{¿Hay
        manteca?}}

        \item \textbf{\textit{¿Hay huevos?} y \textit{¿Hay aceite?}}

        \item \textbf{\textit{¿Hay huevos?} y \textit{¿Hay agua?}}

        \item
            \begin{itemize}[label=, leftmargin=0mm,itemsep=8pt]
                \item  Acá comienza a volverse más divertido. El truco consiste en
                reutilizar preguntas para las cuales ya dimos una equivalencia en
                términos de las base, como \textit{¿Hay para hacer huevos fritos?}.
                De esta forma, la mejor respuesta sería:
                \item \textbf{\textit{¿Hay para hacer huevos fritos?} o \textit{¿Hay
                para hacer huevos duros?}}
                \item Sin embargo es probable que no sea la respuesta de todos en
                una primer instancia. La primer aproximación tiende a ser emplear
                únicamente preguntas base, y por tanto se obtienen respuestas como:
                \item \textbf{(\textit{¿Hay huevos?} y \textit{¿Hay aceite?}) o
                (\textit{¿Hay huevos?} y \textit{¿Hay agua?})}
                \item O también, quienes se percatan de que siemre deben haber
                huevos, que realizan algo como:
                \item \textbf{\textit{¿Hay huevos?} y (\textit{¿Hay aceite?} o
                \textit{¿Hay agua?})}
                \item Luego de explicar que la mejor opción es reutilizar las
                preguntas ya analizadas, se espera respondan las siguientes siempre
                reutilizando.
            \end{itemize}

        \item
            \begin{itemize}[label=, leftmargin=0mm,itemsep=8pt]
                \item Este punto tiene la particularidad de que combinamos una
                pregunta ya realizada en otro ejercicio con una pregunta base. Es
                perfectamente posible hacer esto ya que todas son preguntas que
                portan un valor de verdad.
                \item \textbf{\textit{¿Hay para hacer una torta?} y \textit{¿Hay
                chocolate?}}
            \end{itemize}

        \item
            \begin{itemize}[label=, leftmargin=0mm,itemsep=8pt]
                \item En este caso es conveniente primero definir una nueva
                pregunta, ya que la idea de que se pueda tomar mate, es algo
                independiente de si se puede tomarlo amargo o no. Conviene entonces
                desglosar la pregunta, que de otra forma resulta compleja y con
                muchas partes, en preguntas más pequeñas. El resultado sería:
                \item \textbf{\textit{¿Hay mate?} = \textit{¿Hay yerba?} y
                \textit{¿Hay agua?}}
                \item Y la pregunta original que queríamos definir sería entonces:
                \item \textbf{\textit{¿Hay mate?} y no \textit{¿Hay azúcar?}}
            \end{itemize}

        \item
            \begin{itemize}[label=, leftmargin=0mm,itemsep=8pt]
                \item Acá surge la ventaja de haber hecho \textit{¿Hay mate?}, ya
                que podemos volver a usarla en este punto, resultando en:
                \item \textbf{\textit{¿Hay mate?} y (no \textit{¿Hay para hacer una
                torta de chocolate?} y no \textit{¿Hay para hacer una torta?})}
                \item Aunque, como si no hay para una torta común, tampoco hay para
                una de chocolate, varios podrían percatarse de que basta con:
                \item \textbf{\textit{¿Hay mate?} y no \textit{¿Hay para hacer una
                torta?}}
        \end{itemize}
    \end{enumerate}
\end{solution}

\begin{exercise}
    Considerando las siguientes preguntas básicas y sus respuestas asociadas:
    \begin{itemize}
        \item ¿Plutón es un planeta? $\lfalsefull$
        \item ¿Marte es un planeta?  $\ltruefull$
        \item ¿Marte es un satélite? $\lfalsefull$
        \item ¿Deimos es un satélite?  $\ltruefull$
        \item ¿Ganímedes es un satélite?  $\ltruefull$
        \item ¿Eros es un planeta? $\lfalsefull$
        \item ¿Eros es un satélite? $\lfalsefull$
    \end{itemize}

    Se le pide que, utilizando las mencionadas como preguntas básicas y las
    conectivas lógicas vistas, exprese las preguntas generales a continuación y
    determine el valor de verdad de cada una.
    \begin{enumerate}[a)]
        \item \textbf{¿Son Marte y Plutón planetas?}
        \item \textbf{¿Es Marte un planeta o es Plutón un planeta?}
        \item \textbf{¿Es Marte un planeta o un satélite?}
        \item \textbf{¿Es cierto que Marte es un planeta y Plutón no lo es?}
        \item \textbf{¿Es cierto que Ganímedes y Eros son satélites?}
        \item \textbf{¿Es cierto que Eros no es un satélite, pero Deimos si lo
        es?}
        \item \textbf{¿Es Deimos un satélito o es cierto que Marte es un
        planeta?}
        \item \textbf{¿Es Eros un satélite o es cierto que es un planeta?}
    \end{enumerate}
\end{exercise}

\begin{solution}
    \begin{enumerate}[a)]
        \item
            \begin{itemize}[label=, leftmargin=0mm,itemsep=8pt]
                \item \textbf{¿Son Marte y Plutón planetas?} \lfalsefull
                \item Para determinarlo que es falso, la forma de solución es identificar que la pregunta dada es equivalente a:
                \item \textbf{¿Marte es un planeta? $\land$ ¿Plutón es un planeta?}
                \item Y por tanto, considerando las respuestas a las preguntas base que tenemos, es equivalente a:
                \item \ltruefull ~ $\land$ ~ \lfalsefull
                \item De esta forma, y por la semántica de la conjunción, sabemos que si uno de los elementos es falso, la conjunción completa es falsa. Por lo que se obtiene que el resultado es:
                \item \lfalsefull
                \item En los siguientes items presentaremos el resultado y el análisis línea a línea, sin explicación detallada de los pasos, pero todos siguen el mismo proceso.
            \end{itemize}
        \item
            \begin{itemize}[label=, leftmargin=0mm,itemsep=8pt]
                \item \textbf{¿Es Marte un planeta o es Plutón un planeta?}
            \end{itemize}
        \item \textbf{¿Es Marte un planeta o un satélite?}
        \item \textbf{¿Es cierto que Marte es un planeta y Plutón no lo es?}
        \item \textbf{¿Es cierto que Ganímedes y Eros son satélites?}
        \item \textbf{¿Es cierto que Eros no es un satélite, pero Deimos si lo
        es?}
        \item \textbf{¿Es Deimos un satélito o es cierto que Marte es un
        planeta?}
        \item \textbf{¿Es Eros un satélite o es cierto que es un planeta?}
    \end{enumerate}
\end{solution}

\begin{exercise}
    Considerando las siguientes preguntas como básicas:

    \begin{minipage}{0.45\textwidth}
        \begin{itemize}
            \item ¿hay ejercito enemigo al Norte?
            \item ¿hay ejercito enemigo al Este?
            \item ¿hay ejercito enemigo al Sur?
            \item ¿hay ejercito enemigo al Oeste?
        \end{itemize}
    \end{minipage}
    \begin{minipage}{0.45\textwidth}
        \begin{itemize}
            \item ¿hay ejercito aliado al Norte?
            \item ¿hay ejercito aliado al Este?
            \item ¿hay ejercito aliado al Sur?
            \item ¿hay ejercito aliado al Oeste?
        \end{itemize}
    \end{minipage}

    Se le pide que, utilizando las mencionadas como preguntas básicas\footnote{
        Recuerde que puede elaborar preguntas auxiliares que le ayuden a
        solucionar las pedidas } y las conectivas lógicas vistas, exprese las
        preguntas generales a continuación\footnote{ Se espera escriba algo de
        la forma ``\textit{¿hay manzanas?} $\land$ \textit{¿hay bananas?}'' }:
    \begin{enumerate}[a)]
        \item \textbf{¿Se está amenazado?} (Cuando hay ejercito enemigo en
        alguna dirección)
        \item \textbf{¿Se está libre de peligro?} (Cuando no hay ejercitos
        enemigos en ninguna dirección)
        \item \textbf{¿Se tiene apoyo?} (Cuando hay un ejercito aliado en alguna
        dirección)
        \item \textbf{¿Se está hasta las manos?} (Cuando no hay apoyo y se está
        amenazado)
        \item \textbf{¿Se puede neutralizar alguna amenaza?} (Cuando hay un
        ejercito enemigo en alguna dirección, pero también hay un ejercito
        aliado allí)
        \item \textbf{¿Se puede neutralizar todas las amenazas?} (Cuando se
        puede neutralizar en todas las direcciones)
    \end{enumerate}
\end{exercise}

\begin{exercise}
    Considerando las siguientes preguntas símples:
    \begin{itemize}
        \item ¿Se cumple P?
        \item ¿Se cumple Q?
        \item ¿Se cumple R?
    \end{itemize}

    Se le pide que analice que valuaciones darán verdadero y cuales falso para
        las siguientes preguntas compuestas\footnote{ Si pensar en preguntas
        como ``\textit{¿Se cumple P?}'' le resulta confuso, piense en
        reemplazarlas por preguntas que le sean más familiares, como
        ``\textit{¿Se cumple que el lobo vive en el bosque?}''. Lo interesante
        del ejercicio es que la pregunta en cuestión no es realmente relevante,
        sino la forma de la pregunta compuesta. }. Además se le pide determine
        cuales de ellas se tratan de tautologías, cuales de contradicciones y
        cuáles de contingencias.
    \begin{enumerate}[a)]
        \item \textbf{¿Se cumple P? $\land$ $\lnot$ ¿Se cumple P?}
        \item \textbf{¿Se cumple P? $\lor$ $\lnot$ ¿Se cumple P?}
        \item \textbf{¿Se cumple P? $\land$ ¿Se cumple Q?}
        \item \textbf{¿Se cumple P? $\land$ (¿Se cumple Q? $\lor$ ¿Se cumple
        R?)}
        \item \textbf{¿Se cumple P? $\lor$ ¿Se cumple Q?}
        \item \textbf{$\lnot$ ¿Se cumple P? $\lor$ (¿Se cumple Q? $\land$ ¿Se
        cumple R?)}
        \item \textbf{$\lnot$ ¿Se cumple P? $\lor$ $\lnot$ (¿Se cumple Q? $\lor$
        ¿Se cumple R?)}
    \end{enumerate}
\end{exercise}

\begin{exercise}
    Sabiendo que las siguientes preguntas compuestas evalúan todas a
    $\ltruefull$
    \begin{itemize}
    \item ¿La palmera está creciendo torcida? y ¿Es cierto que el árbol no dio
    paltas este año?
    \item ¿Es cierto que no hay flores en el cantero? o ¿El árbol dio paltas
    este año?
    \end{itemize}

    Se pide que responda las siguientes preguntas simples
    \begin{enumerate}[a)]
        \item ¿El árbol dio paltas este año?
        \item ¿Hay flores en el cantero?
        \item ¿La palmera está creciendo torcida?
    \end{enumerate}
\end{exercise}

\begin{exercise}
    Dadas las siguientes preguntas base y sus respuestas:
    \begin{itemize}
        \item ¿Es Microsoft el creador de Windows? $\ltruefull$
        \item ¿Es Microsoft el creador de GNU? $\lfalsefull$
        \item ¿Es Richard Stallman el creador de Windows? $\lfalsefull$
        \item ¿Es Richard Stallman el creador de GNU? $\ltruefull$
    \end{itemize}

    Analice las preguntas a continuación, detecte las conectivas involucradas y
    reemplantee las preguntas para que estén elaboradas en términos de las
    preguntas dadas. Luego, analice el valor de verdad de las mismas.

    \begin{enumerate}[a)]
        \item ¿Es cierto que no es cierto que Microsoft es el creador de GNU?
        \item ¿Es cierto que Richard Stallman es el creador de Windows o de GNU?
        \item ¿Es cierto que Microsoft no es el creador de GNU pero que Richard
        Stallman si lo es?
        \item ¿Es cierto que GNU no es del mismo creador que Windows?
    \end{enumerate}
\end{exercise}

\begin{exercise}
    Se ha encontrado vida en otro planeta, y se ha decidido nombrar a los
    animales encontrados como ``Woofle'', ``Brlfks'' y ``Morlock''.

    Cada animal tiene sus características distintivas (pueden ser grandes o
    chicos, con o sin pelo, acuáticos o terrestres).

    Si las respuestas a todas las preguntas siguientes son $\ltruefull$, enumere
    que características tiene cada animal. Para ello, realice la tabla de verdad
    de cada pregunta y analice las respuestas a las preguntas base en las
    valuaciones verdaderas.

    \begin{itemize}
      \item ¿El Woofle tiene pelo y no es acuático?
      \item ¿El Morlock es terrestre y es grande, y el Brlfks es acuático?
      \item ¿El Brlfks es pequeño o el Morlock es pequeño?
      \item ¿El Woofle no es grande o el Brlfks es grande? ¿Y es cierto que el
      Morlock no es pequeño?
      \item ¿El Woofle tiene pelo y el Brlfks no? ¿O es cierto que el Brlfks no
      tiene pelo y el Morlock tiene pelo?
    \end{itemize}
\end{exercise}