\chapterimage{capitulos/logica/imagenes/cover}
\chapterimagedescription{Red Lógica de un Enrutador con LEDs}
\chapterimageauthor{Fotografía de tschoenemeyer}

\chapter{Lógica}
\index{Logica@Lógica}

Este primer capítulo trata sobre Lógica. La palabra ``lógica''
deriva del griego antiguo $\lambda o \gamma \iota \kappa \eta$
%λογική
(logiké), que significa ``dotada
de razón, intelectual, argumentativa''  y que a su vez
viene de $\lambda o \gamma o \varsigma$ (lógos), ``palabra, pensamiento, idea, argumento,
razón o principio''.

La lógica nace en la antiguedad, al rededor del siglo IV a.C.,
en forma independiente tanto en China, como en la India y Grecia.
Es la lógica de esta última civilización la que realizaría un
tratamiento más formal, y que posteriormente sería trabajado por
los lógicos islámicos, y posteriormente por los lógicos de la Edad
Media.

La lógica es una ciencia formal que estudia los principios de la demostración
y la inferencia válida, las falacias, las paradojas y la noción de verdad.
Corresponde tanto a una rama de la filosofía como a las matemáticas, según el enfoque, y se considera una herramienta básica para todas las ciencias.

Aristóteles fue el primero en formalizar los razonamientos, utilizando letras para representar términos. También fue el primero en emplear el término ``lógica'' para referirse al estudio de los argumentos dentro del ``lenguaje apofántico'' como manifestador de la verdad en la ciencia. Sostuvo que la verdad se manifiesta en el juicio verdadero y el argumento válido en un silogismo (Silogismo es un argumento en el cual, establecidas ciertas cosas, resulta necesariamente de ellas, por ser lo que son, otra cosa diferente).  Aristóteles también formalizó el cuadro de oposición de los juicios y categorizó las formas válidas del silogismo. Además, Aristóteles reconoció y estudió los argumentos inductivos, base de lo que constituye la ciencia experimental, cuya lógica está estrechamente ligada al método científico.

La lógica aristotélica, basada principalmente en el análisis del lenguaje
natural, permaneció como el principal método de análisis lógico hasta entrado el siglo XVII cuando filósofos y matemáticos como Descartes o Leibniz plantearon la idea de un lenguaje universal, capaz de expresar todos los conceptos matemáticos y de la razón. Esta idea requería la formalización de
los procesos de razonamiento, y el desarrollo de una sintaxis precisa
que permita el desarrollo de un cálculo para analizar los razonamientos. Esta
premisa es la base fundamental de los desarrollos lógicos de la edad moderna.

Gottlob Frege en su Begriffsschrift (1879) extendió la lógica formal más allá
de la lógica proposicional para incluir constructores como ``todo'' y 
``algunos''. Mostró cómo introducir variables y cuantificadores para revelar
la estructura lógica de las oraciones, que podría estar ocultas tras su 
estructura gramatical. Pocos años después matemáticos como Giuseppe Peano,
Ernst Schröder y Charles Peirce introdujo el término "Lógica de segundo orden" 
proporcionando la mayor parte de la moderna notación lógica. En 1847, George 
Boole publicó un breve tratado titulado ``El análisis matemático de la 
lógica'', y en 1854 otro más importante titulado ``Las leyes del 
pensamiento''. La idea de Boole fue construir a la lógica como un cálculo con 
propiedades similares a la matemática. Al mismo tiempo, Augustus De Morgan 
publica en 1847 su obra ``Lógica formal'', donde introduce las ``leyes de De 
Morgan'' e intenta generalizar la noción de silogismo. Otro importante 
contribuyente inglés fue John Venn, quien en 1881 publicó su libro ``Lógica 
Simbólica'', donde introdujo los famosos diagramas de Venn.

En 1910, Bertrand Russell y Alfred North Whitehead publican ``Principia 
mathematica'', un trabajo en el que logran construir gran parte de la 
matemática a partir de la lógica, demostrando que la lógica es una ciencia
subyacente a las matemáticas.

Más de 2000 años de historia nos llevan a la actualidad de la lógica
como ciencia, que hoy sustenta cualquier desarrollo científico, tanto de las ciencias empíricas (biología, química, física, etc.) como formales (matemática y sus diversas ramas) y también las ciencias sociales (legales, comunicación, filosofía, etc.)

Hoy en día se entiende a la lógica como una ciencia que requiere un análisis formal (parecido a las matemáticas) y que sigue métodos estrictos para su análisis.

Existen diversos típos de lógicas, como la lógica proposicional, la lógica de 
predicados, logica hermitica, logicas modales, entre otras varias. Nosotros nos
centraremos en la lógica proposicional y la lógica de predicados.

Por otro lado, se pueden analizar diversos tipos de razonamientos. Por ejemplo,
los razonamientos inductivos, ya sea por analogía o por enumeración. A 
nosotros nos interesarán exclusivamente los razonamientos deductivos, pues 
poseen una característica fundamental que requieren las ciencias formales, 
como son las ciencias de la computación.

\section{Razonamientos Deductivos y la Lógica}

Para poder comprender la lógica como ciencia formal, deberemos primero 
establecer algunas definiciones básicas. Recurriremos luego a la intuición 
para analizar razonamientos. Posteriormente veremos como realizar una análisis 
formal de dichos razonamientos.

